
% Default to the notebook output style




% Inherit from the specified cell style.





\documentclass{article}



    \usepackage{graphicx} % Used to insert images
    \usepackage{adjustbox} % Used to constrain images to a maximum size
    \usepackage{color} % Allow colors to be defined
    \usepackage{enumerate} % Needed for markdown enumerations to work
    \usepackage{geometry} % Used to adjust the document margins
    \usepackage{amsmath} % Equations
    \usepackage{amssymb} % Equations
    \usepackage{eurosym} % defines \euro
    \usepackage[mathletters]{ucs} % Extended unicode (utf-8) support
    \usepackage[utf8x]{inputenc} % Allow utf-8 characters in the tex document
    \usepackage{fancyvrb} % verbatim replacement that allows latex
    \usepackage{grffile} % extends the file name processing of package graphics
                         % to support a larger range
    % The hyperref package gives us a pdf with properly built
    % internal navigation ('pdf bookmarks' for the table of contents,
    % internal cross-reference links, web links for URLs, etc.)
    \usepackage{hyperref}
    \usepackage{longtable} % longtable support required by pandoc >1.10
    \usepackage{booktabs}  % table support for pandoc > 1.12.2




    \definecolor{orange}{cmyk}{0,0.4,0.8,0.2}
    \definecolor{darkorange}{rgb}{.71,0.21,0.01}
    \definecolor{darkgreen}{rgb}{.12,.54,.11}
    \definecolor{myteal}{rgb}{.26, .44, .56}
    \definecolor{gray}{gray}{0.45}
    \definecolor{lightgray}{gray}{.95}
    \definecolor{mediumgray}{gray}{.8}
    \definecolor{inputbackground}{rgb}{.95, .95, .85}
    \definecolor{outputbackground}{rgb}{.95, .95, .95}
    \definecolor{traceback}{rgb}{1, .95, .95}
    % ansi colors
    \definecolor{red}{rgb}{.6,0,0}
    \definecolor{green}{rgb}{0,.65,0}
    \definecolor{brown}{rgb}{0.6,0.6,0}
    \definecolor{blue}{rgb}{0,.145,.698}
    \definecolor{purple}{rgb}{.698,.145,.698}
    \definecolor{cyan}{rgb}{0,.698,.698}
    \definecolor{lightgray}{gray}{0.5}

    % bright ansi colors
    \definecolor{darkgray}{gray}{0.25}
    \definecolor{lightred}{rgb}{1.0,0.39,0.28}
    \definecolor{lightgreen}{rgb}{0.48,0.99,0.0}
    \definecolor{lightblue}{rgb}{0.53,0.81,0.92}
    \definecolor{lightpurple}{rgb}{0.87,0.63,0.87}
    \definecolor{lightcyan}{rgb}{0.5,1.0,0.83}

    % commands and environments needed by pandoc snippets
    % extracted from the output of `pandoc -s`
    \DefineVerbatimEnvironment{Highlighting}{Verbatim}{commandchars=\\\{\}}
    % Add ',fontsize=\small' for more characters per line
    \newenvironment{Shaded}{}{}
    \newcommand{\KeywordTok}[1]{\textcolor[rgb]{0.00,0.44,0.13}{\textbf{{#1}}}}
    \newcommand{\DataTypeTok}[1]{\textcolor[rgb]{0.56,0.13,0.00}{{#1}}}
    \newcommand{\DecValTok}[1]{\textcolor[rgb]{0.25,0.63,0.44}{{#1}}}
    \newcommand{\BaseNTok}[1]{\textcolor[rgb]{0.25,0.63,0.44}{{#1}}}
    \newcommand{\FloatTok}[1]{\textcolor[rgb]{0.25,0.63,0.44}{{#1}}}
    \newcommand{\CharTok}[1]{\textcolor[rgb]{0.25,0.44,0.63}{{#1}}}
    \newcommand{\StringTok}[1]{\textcolor[rgb]{0.25,0.44,0.63}{{#1}}}
    \newcommand{\CommentTok}[1]{\textcolor[rgb]{0.38,0.63,0.69}{\textit{{#1}}}}
    \newcommand{\OtherTok}[1]{\textcolor[rgb]{0.00,0.44,0.13}{{#1}}}
    \newcommand{\AlertTok}[1]{\textcolor[rgb]{1.00,0.00,0.00}{\textbf{{#1}}}}
    \newcommand{\FunctionTok}[1]{\textcolor[rgb]{0.02,0.16,0.49}{{#1}}}
    \newcommand{\RegionMarkerTok}[1]{{#1}}
    \newcommand{\ErrorTok}[1]{\textcolor[rgb]{1.00,0.00,0.00}{\textbf{{#1}}}}
    \newcommand{\NormalTok}[1]{{#1}}

    % Define a nice break command that doesn't care if a line doesn't already
    % exist.
    \def\br{\hspace*{\fill} \\* }
    % Math Jax compatability definitions
    \def\gt{>}
    \def\lt{<}
    % Document parameters
    \title{Orbital Insights Data Analysis}
    \author{Daniel Cuneo \\
            2258 Jones Street \\
            San Francisco CA 94133 \\
            415-871-1909 \\
            dpcuneo@fastmail.fm \\
            }


    % Pygments definitions

\makeatletter
\def\PY@reset{\let\PY@it=\relax \let\PY@bf=\relax%
    \let\PY@ul=\relax \let\PY@tc=\relax%
    \let\PY@bc=\relax \let\PY@ff=\relax}
\def\PY@tok#1{\csname PY@tok@#1\endcsname}
\def\PY@toks#1+{\ifx\relax#1\empty\else%
    \PY@tok{#1}\expandafter\PY@toks\fi}
\def\PY@do#1{\PY@bc{\PY@tc{\PY@ul{%
    \PY@it{\PY@bf{\PY@ff{#1}}}}}}}
\def\PY#1#2{\PY@reset\PY@toks#1+\relax+\PY@do{#2}}

\expandafter\def\csname PY@tok@gd\endcsname{\def\PY@tc##1{\textcolor[rgb]{0.63,0.00,0.00}{##1}}}
\expandafter\def\csname PY@tok@gu\endcsname{\let\PY@bf=\textbf\def\PY@tc##1{\textcolor[rgb]{0.50,0.00,0.50}{##1}}}
\expandafter\def\csname PY@tok@gt\endcsname{\def\PY@tc##1{\textcolor[rgb]{0.00,0.27,0.87}{##1}}}
\expandafter\def\csname PY@tok@gs\endcsname{\let\PY@bf=\textbf}
\expandafter\def\csname PY@tok@gr\endcsname{\def\PY@tc##1{\textcolor[rgb]{1.00,0.00,0.00}{##1}}}
\expandafter\def\csname PY@tok@cm\endcsname{\let\PY@it=\textit\def\PY@tc##1{\textcolor[rgb]{0.25,0.50,0.50}{##1}}}
\expandafter\def\csname PY@tok@vg\endcsname{\def\PY@tc##1{\textcolor[rgb]{0.10,0.09,0.49}{##1}}}
\expandafter\def\csname PY@tok@m\endcsname{\def\PY@tc##1{\textcolor[rgb]{0.40,0.40,0.40}{##1}}}
\expandafter\def\csname PY@tok@mh\endcsname{\def\PY@tc##1{\textcolor[rgb]{0.40,0.40,0.40}{##1}}}
\expandafter\def\csname PY@tok@go\endcsname{\def\PY@tc##1{\textcolor[rgb]{0.53,0.53,0.53}{##1}}}
\expandafter\def\csname PY@tok@ge\endcsname{\let\PY@it=\textit}
\expandafter\def\csname PY@tok@vc\endcsname{\def\PY@tc##1{\textcolor[rgb]{0.10,0.09,0.49}{##1}}}
\expandafter\def\csname PY@tok@il\endcsname{\def\PY@tc##1{\textcolor[rgb]{0.40,0.40,0.40}{##1}}}
\expandafter\def\csname PY@tok@cs\endcsname{\let\PY@it=\textit\def\PY@tc##1{\textcolor[rgb]{0.25,0.50,0.50}{##1}}}
\expandafter\def\csname PY@tok@cp\endcsname{\def\PY@tc##1{\textcolor[rgb]{0.74,0.48,0.00}{##1}}}
\expandafter\def\csname PY@tok@gi\endcsname{\def\PY@tc##1{\textcolor[rgb]{0.00,0.63,0.00}{##1}}}
\expandafter\def\csname PY@tok@gh\endcsname{\let\PY@bf=\textbf\def\PY@tc##1{\textcolor[rgb]{0.00,0.00,0.50}{##1}}}
\expandafter\def\csname PY@tok@ni\endcsname{\let\PY@bf=\textbf\def\PY@tc##1{\textcolor[rgb]{0.60,0.60,0.60}{##1}}}
\expandafter\def\csname PY@tok@nl\endcsname{\def\PY@tc##1{\textcolor[rgb]{0.63,0.63,0.00}{##1}}}
\expandafter\def\csname PY@tok@nn\endcsname{\let\PY@bf=\textbf\def\PY@tc##1{\textcolor[rgb]{0.00,0.00,1.00}{##1}}}
\expandafter\def\csname PY@tok@no\endcsname{\def\PY@tc##1{\textcolor[rgb]{0.53,0.00,0.00}{##1}}}
\expandafter\def\csname PY@tok@na\endcsname{\def\PY@tc##1{\textcolor[rgb]{0.49,0.56,0.16}{##1}}}
\expandafter\def\csname PY@tok@nb\endcsname{\def\PY@tc##1{\textcolor[rgb]{0.00,0.50,0.00}{##1}}}
\expandafter\def\csname PY@tok@nc\endcsname{\let\PY@bf=\textbf\def\PY@tc##1{\textcolor[rgb]{0.00,0.00,1.00}{##1}}}
\expandafter\def\csname PY@tok@nd\endcsname{\def\PY@tc##1{\textcolor[rgb]{0.67,0.13,1.00}{##1}}}
\expandafter\def\csname PY@tok@ne\endcsname{\let\PY@bf=\textbf\def\PY@tc##1{\textcolor[rgb]{0.82,0.25,0.23}{##1}}}
\expandafter\def\csname PY@tok@nf\endcsname{\def\PY@tc##1{\textcolor[rgb]{0.00,0.00,1.00}{##1}}}
\expandafter\def\csname PY@tok@si\endcsname{\let\PY@bf=\textbf\def\PY@tc##1{\textcolor[rgb]{0.73,0.40,0.53}{##1}}}
\expandafter\def\csname PY@tok@s2\endcsname{\def\PY@tc##1{\textcolor[rgb]{0.73,0.13,0.13}{##1}}}
\expandafter\def\csname PY@tok@vi\endcsname{\def\PY@tc##1{\textcolor[rgb]{0.10,0.09,0.49}{##1}}}
\expandafter\def\csname PY@tok@nt\endcsname{\let\PY@bf=\textbf\def\PY@tc##1{\textcolor[rgb]{0.00,0.50,0.00}{##1}}}
\expandafter\def\csname PY@tok@nv\endcsname{\def\PY@tc##1{\textcolor[rgb]{0.10,0.09,0.49}{##1}}}
\expandafter\def\csname PY@tok@s1\endcsname{\def\PY@tc##1{\textcolor[rgb]{0.73,0.13,0.13}{##1}}}
\expandafter\def\csname PY@tok@kd\endcsname{\let\PY@bf=\textbf\def\PY@tc##1{\textcolor[rgb]{0.00,0.50,0.00}{##1}}}
\expandafter\def\csname PY@tok@sh\endcsname{\def\PY@tc##1{\textcolor[rgb]{0.73,0.13,0.13}{##1}}}
\expandafter\def\csname PY@tok@sc\endcsname{\def\PY@tc##1{\textcolor[rgb]{0.73,0.13,0.13}{##1}}}
\expandafter\def\csname PY@tok@sx\endcsname{\def\PY@tc##1{\textcolor[rgb]{0.00,0.50,0.00}{##1}}}
\expandafter\def\csname PY@tok@bp\endcsname{\def\PY@tc##1{\textcolor[rgb]{0.00,0.50,0.00}{##1}}}
\expandafter\def\csname PY@tok@c1\endcsname{\let\PY@it=\textit\def\PY@tc##1{\textcolor[rgb]{0.25,0.50,0.50}{##1}}}
\expandafter\def\csname PY@tok@kc\endcsname{\let\PY@bf=\textbf\def\PY@tc##1{\textcolor[rgb]{0.00,0.50,0.00}{##1}}}
\expandafter\def\csname PY@tok@c\endcsname{\let\PY@it=\textit\def\PY@tc##1{\textcolor[rgb]{0.25,0.50,0.50}{##1}}}
\expandafter\def\csname PY@tok@mf\endcsname{\def\PY@tc##1{\textcolor[rgb]{0.40,0.40,0.40}{##1}}}
\expandafter\def\csname PY@tok@err\endcsname{\def\PY@bc##1{\setlength{\fboxsep}{0pt}\fcolorbox[rgb]{1.00,0.00,0.00}{1,1,1}{\strut ##1}}}
\expandafter\def\csname PY@tok@mb\endcsname{\def\PY@tc##1{\textcolor[rgb]{0.40,0.40,0.40}{##1}}}
\expandafter\def\csname PY@tok@ss\endcsname{\def\PY@tc##1{\textcolor[rgb]{0.10,0.09,0.49}{##1}}}
\expandafter\def\csname PY@tok@sr\endcsname{\def\PY@tc##1{\textcolor[rgb]{0.73,0.40,0.53}{##1}}}
\expandafter\def\csname PY@tok@mo\endcsname{\def\PY@tc##1{\textcolor[rgb]{0.40,0.40,0.40}{##1}}}
\expandafter\def\csname PY@tok@kn\endcsname{\let\PY@bf=\textbf\def\PY@tc##1{\textcolor[rgb]{0.00,0.50,0.00}{##1}}}
\expandafter\def\csname PY@tok@mi\endcsname{\def\PY@tc##1{\textcolor[rgb]{0.40,0.40,0.40}{##1}}}
\expandafter\def\csname PY@tok@gp\endcsname{\let\PY@bf=\textbf\def\PY@tc##1{\textcolor[rgb]{0.00,0.00,0.50}{##1}}}
\expandafter\def\csname PY@tok@o\endcsname{\def\PY@tc##1{\textcolor[rgb]{0.40,0.40,0.40}{##1}}}
\expandafter\def\csname PY@tok@kr\endcsname{\let\PY@bf=\textbf\def\PY@tc##1{\textcolor[rgb]{0.00,0.50,0.00}{##1}}}
\expandafter\def\csname PY@tok@s\endcsname{\def\PY@tc##1{\textcolor[rgb]{0.73,0.13,0.13}{##1}}}
\expandafter\def\csname PY@tok@kp\endcsname{\def\PY@tc##1{\textcolor[rgb]{0.00,0.50,0.00}{##1}}}
\expandafter\def\csname PY@tok@w\endcsname{\def\PY@tc##1{\textcolor[rgb]{0.73,0.73,0.73}{##1}}}
\expandafter\def\csname PY@tok@kt\endcsname{\def\PY@tc##1{\textcolor[rgb]{0.69,0.00,0.25}{##1}}}
\expandafter\def\csname PY@tok@ow\endcsname{\let\PY@bf=\textbf\def\PY@tc##1{\textcolor[rgb]{0.67,0.13,1.00}{##1}}}
\expandafter\def\csname PY@tok@sb\endcsname{\def\PY@tc##1{\textcolor[rgb]{0.73,0.13,0.13}{##1}}}
\expandafter\def\csname PY@tok@k\endcsname{\let\PY@bf=\textbf\def\PY@tc##1{\textcolor[rgb]{0.00,0.50,0.00}{##1}}}
\expandafter\def\csname PY@tok@se\endcsname{\let\PY@bf=\textbf\def\PY@tc##1{\textcolor[rgb]{0.73,0.40,0.13}{##1}}}
\expandafter\def\csname PY@tok@sd\endcsname{\let\PY@it=\textit\def\PY@tc##1{\textcolor[rgb]{0.73,0.13,0.13}{##1}}}

\def\PYZbs{\char`\\}
\def\PYZus{\char`\_}
\def\PYZob{\char`\{}
\def\PYZcb{\char`\}}
\def\PYZca{\char`\^}
\def\PYZam{\char`\&}
\def\PYZlt{\char`\<}
\def\PYZgt{\char`\>}
\def\PYZsh{\char`\#}
\def\PYZpc{\char`\%}
\def\PYZdl{\char`\$}
\def\PYZhy{\char`\-}
\def\PYZsq{\char`\'}
\def\PYZdq{\char`\"}
\def\PYZti{\char`\~}
% for compatibility with earlier versions
\def\PYZat{@}
\def\PYZlb{[}
\def\PYZrb{]}
\makeatother


    % Exact colors from NB
    \definecolor{incolor}{rgb}{0.0, 0.0, 0.5}
    \definecolor{outcolor}{rgb}{0.545, 0.0, 0.0}




    % Prevent overflowing lines due to hard-to-break entities
    \sloppy
    % Setup hyperref package
    \hypersetup{
      breaklinks=true,  % so long urls are correctly broken across lines
      colorlinks=true,
      urlcolor=blue,
      linkcolor=darkorange,
      citecolor=darkgreen,
      }
    % Slightly bigger margins than the latex defaults

    \geometry{verbose,tmargin=1in,bmargin=1in,lmargin=1in,rmargin=1in}



    \begin{document}


    \maketitle






    \begin{Verbatim}[commandchars=\\\{\}]
{\color{incolor}In [{\color{incolor}1}]:} \PY{o}{\PYZpc{}\PYZpc{}}\PY{k}{bash}
        date
\end{Verbatim}

    \begin{Verbatim}[commandchars=\\\{\}]
Mon Mar 14 15:01:04 PDT 2016
    \end{Verbatim}

    \begin{Verbatim}[commandchars=\\\{\}]
{\color{incolor}In [{\color{incolor}2}]:} \PY{c}{\PYZsh{} https://github.com/rasbt/watermark}
        \PY{o}{\PYZpc{}}\PY{k}{load\PYZus{}ext} watermark
        \PY{o}{\PYZpc{}}\PY{k}{watermark} \PYZhy{}p numpy,scipy,pandas,matplotlib
\end{Verbatim}

    \begin{Verbatim}[commandchars=\\\{\}]
The watermark extension is already loaded. To reload it, use:
  \%reload\_ext watermark

numpy 1.10.1
scipy 0.16.0
pandas 0.16.2
matplotlib 1.4.0
    \end{Verbatim}

    \section{First Look}\label{first-look}

    \subsection{Shell}\label{shell}

    \begin{Verbatim}[commandchars=\\\{\}]
{\color{incolor}In [{\color{incolor}2}]:} \PY{o}{\PYZpc{}\PYZpc{}}\PY{k}{bash}
        root=\PYZdq{}/home/daniel/OrbitalInsights\PYZdq{}
        cat \PYZdl{}root/data.csv | wc \PYZhy{}l
\end{Verbatim}

    \begin{Verbatim}[commandchars=\\\{\}]
1827
    \end{Verbatim}

    1826 rows of data

    \begin{Verbatim}[commandchars=\\\{\}]
{\color{incolor}In [{\color{incolor}6}]:} \PY{o}{\PYZpc{}\PYZpc{}}\PY{k}{bash}
        root=\PYZdq{}/home/daniel/OrbitalInsights\PYZdq{}
        head \PYZhy{}n 2 \PYZdl{}root/data.csv
        tail \PYZhy{}n 1 \PYZdl{}root/data.csv
\end{Verbatim}

    \begin{Verbatim}[commandchars=\\\{\}]
date,day.of.week,car.count,weather
2010-01-01,friday,94.5,-0.1
2014-12-31,wednesday,166,-0.2
    \end{Verbatim}

    Data set for 5 years, Jan 1, 2010 to Dec 31 2014

    \begin{Verbatim}[commandchars=\\\{\}]
{\color{incolor}In [{\color{incolor}9}]:} \PY{l+m+mi}{1826} \PY{o}{/} \PY{l+m+mf}{5.0} \PY{c}{\PYZsh{} guess sample rate}
\end{Verbatim}

            \begin{Verbatim}[commandchars=\\\{\}]
{\color{outcolor}Out[{\color{outcolor}9}]:} 365.2
\end{Verbatim}

    Probably 365 data points per year, one data point per day.

    \subsection{Python}\label{python}

    \subsubsection{Initial Plot}\label{initial-plot}

    \begin{Verbatim}[commandchars=\\\{\}]
{\color{incolor}In [{\color{incolor}2}]:} \PY{k+kn}{import} \PY{n+nn}{matplotlib.pyplot} \PY{k+kn}{as} \PY{n+nn}{plt}
        \PY{k+kn}{import} \PY{n+nn}{numpy} \PY{k+kn}{as} \PY{n+nn}{np}
        \PY{k+kn}{import} \PY{n+nn}{pandas} \PY{k+kn}{as} \PY{n+nn}{pd}
        \PY{o}{\PYZpc{}}\PY{k}{matplotlib} inline
        \PY{k+kn}{import} \PY{n+nn}{scipy.signal} \PY{k+kn}{as} \PY{n+nn}{signal}
\end{Verbatim}

    \begin{Verbatim}[commandchars=\\\{\}]
{\color{incolor}In [{\color{incolor}3}]:} \PY{n}{df} \PY{o}{=} \PY{n}{pd}\PY{o}{.}\PY{n}{read\PYZus{}csv}\PY{p}{(}\PY{l+s}{\PYZdq{}}\PY{l+s}{/home/daniel/OrbitalInsights/data.csv}\PY{l+s}{\PYZdq{}}\PY{p}{)}
\end{Verbatim}

    \begin{Verbatim}[commandchars=\\\{\}]
{\color{incolor}In [{\color{incolor}4}]:} \PY{n}{df}\PY{o}{.}\PY{n}{info}\PY{p}{(}\PY{p}{)}
\end{Verbatim}

    \begin{Verbatim}[commandchars=\\\{\}]
<class 'pandas.core.frame.DataFrame'>
Int64Index: 1826 entries, 0 to 1825
Data columns (total 4 columns):
date           1826 non-null object
day.of.week    1826 non-null object
car.count      1799 non-null float64
weather        1826 non-null float64
dtypes: float64(2), object(2)
memory usage: 71.3+ KB
    \end{Verbatim}

    \begin{Verbatim}[commandchars=\\\{\}]
{\color{incolor}In [{\color{incolor}5}]:} \PY{n}{df}\PY{o}{.}\PY{n}{head}\PY{p}{(}\PY{p}{)}
\end{Verbatim}

            \begin{Verbatim}[commandchars=\\\{\}]
{\color{outcolor}Out[{\color{outcolor}5}]:}          date day.of.week  car.count  weather
        0  2010-01-01      friday       94.5     -0.1
        1  2010-01-02    saturday      108.4     -2.4
        2  2010-01-03      sunday      105.5     -0.5
        3  2010-01-04      monday      109.6     -2.1
        4  2010-01-05     tuesday      116.1      1.9
\end{Verbatim}

    \begin{Verbatim}[commandchars=\\\{\}]
{\color{incolor}In [{\color{incolor}6}]:} \PY{c}{\PYZsh{} like using Pandas b/c of the datetime features, resample or groupby}
        \PY{c}{\PYZsh{} but I haven\PYZsq{}t used date methods in a while and lost some time to redresh my memory}
        \PY{n}{df}\PY{p}{[}\PY{l+s}{\PYZsq{}}\PY{l+s}{date}\PY{l+s}{\PYZsq{}}\PY{p}{]} \PY{o}{=} \PY{n}{pd}\PY{o}{.}\PY{n}{to\PYZus{}datetime}\PY{p}{(}\PY{n}{df}\PY{p}{[}\PY{l+s}{\PYZsq{}}\PY{l+s}{date}\PY{l+s}{\PYZsq{}}\PY{p}{]}\PY{p}{)}
        \PY{n}{df}\PY{o}{.}\PY{n}{set\PYZus{}index}\PY{p}{(}\PY{n}{df}\PY{p}{[}\PY{l+s}{\PYZsq{}}\PY{l+s}{date}\PY{l+s}{\PYZsq{}}\PY{p}{]}\PY{p}{,} \PY{n}{inplace}\PY{o}{=}\PY{n+nb+bp}{True}\PY{p}{)}
\end{Verbatim}

    \begin{Verbatim}[commandchars=\\\{\}]
{\color{incolor}In [{\color{incolor}7}]:} \PY{n}{df}\PY{p}{[}\PY{l+s}{\PYZsq{}}\PY{l+s}{car.count}\PY{l+s}{\PYZsq{}}\PY{p}{]}\PY{o}{.}\PY{n}{plot}\PY{p}{(}\PY{n}{rot}\PY{o}{=}\PY{l+m+mi}{90}\PY{p}{,} \PY{n}{figsize}\PY{o}{=}\PY{p}{(}\PY{l+m+mi}{15}\PY{p}{,}\PY{l+m+mi}{5}\PY{p}{)}\PY{p}{)}
\end{Verbatim}

            \begin{Verbatim}[commandchars=\\\{\}]
{\color{outcolor}Out[{\color{outcolor}7}]:} <matplotlib.axes.\_subplots.AxesSubplot at 0x7fd8f85ba690>
\end{Verbatim}

    \begin{center}
    \adjustimage{max size={0.9\linewidth}{0.9\paperheight}}{analysis_files/analysis_18_1.png}
    \end{center}
    { \hspace*{\fill} \\}

    \paragraph{Roughly quadratic over the 5 years, with more interesting
shorter
trend}\label{roughly-quadratic-over-the-5-years-with-more-interesting-shorter-trend}

    \subsubsection{Single Year Analysis: first year on
record}\label{single-year-analysis-first-year-on-record}

    \begin{Verbatim}[commandchars=\\\{\}]
{\color{incolor}In [{\color{incolor}9}]:} \PY{n}{df}\PY{p}{[}\PY{l+s}{\PYZsq{}}\PY{l+s}{car.count}\PY{l+s}{\PYZsq{}}\PY{p}{]}\PY{p}{[}\PY{l+m+mi}{0}\PY{p}{:}\PY{l+m+mi}{365}\PY{p}{]}\PY{o}{.}\PY{n}{plot}\PY{p}{(}\PY{n}{rot}\PY{o}{=}\PY{l+m+mi}{90}\PY{p}{,} \PY{n}{figsize}\PY{o}{=}\PY{p}{(}\PY{l+m+mi}{15}\PY{p}{,}\PY{l+m+mi}{7}\PY{p}{)}\PY{p}{,} \PY{n}{title}\PY{o}{=}\PY{l+s}{\PYZdq{}}\PY{l+s}{Single Year Analysis: First Year on Record}\PY{l+s}{\PYZdq{}}\PY{p}{)}
\end{Verbatim}

            \begin{Verbatim}[commandchars=\\\{\}]
{\color{outcolor}Out[{\color{outcolor}9}]:} <matplotlib.axes.\_subplots.AxesSubplot at 0x7fd8f1174450>
\end{Verbatim}

    \begin{center}
    \adjustimage{max size={0.9\linewidth}{0.9\paperheight}}{analysis_files/analysis_21_1.png}
    \end{center}
    { \hspace*{\fill} \\}

    Deal with non finite values so we can detrend.

    \begin{Verbatim}[commandchars=\\\{\}]
{\color{incolor}In [{\color{incolor}10}]:} \PY{n}{year} \PY{o}{=} \PY{n}{df}\PY{p}{[}\PY{l+s}{\PYZsq{}}\PY{l+s}{car.count}\PY{l+s}{\PYZsq{}}\PY{p}{]}\PY{p}{[}\PY{l+m+mi}{0}\PY{p}{:}\PY{l+m+mi}{365}\PY{p}{]}
         \PY{n}{year}\PY{p}{[}\PY{o}{\PYZti{}}\PY{n}{np}\PY{o}{.}\PY{n}{isfinite}\PY{p}{(}\PY{n}{year}\PY{p}{)}\PY{p}{]}
\end{Verbatim}

            \begin{Verbatim}[commandchars=\\\{\}]
{\color{outcolor}Out[{\color{outcolor}10}]:} date
         2010-02-09   NaN
         2010-04-12   NaN
         2010-06-17   NaN
         2010-06-22   NaN
         2010-06-23   NaN
         2010-08-20   NaN
         2010-08-31   NaN
         2010-10-04   NaN
         2010-10-05   NaN
         2010-10-07   NaN
         2010-10-10   NaN
         2010-11-10   NaN
         Name: car.count, dtype: float64
\end{Verbatim}

    \begin{Verbatim}[commandchars=\\\{\}]
{\color{incolor}In [{\color{incolor}11}]:} \PY{c}{\PYZsh{} I wasted about 10 minutes going for a vectorized method had to move to brute force}
         \PY{k}{def} \PY{n+nf}{remove\PYZus{}nans}\PY{p}{(}\PY{n}{data}\PY{p}{)}\PY{p}{:}
             \PY{k}{for} \PY{n}{i} \PY{o+ow}{in} \PY{n+nb}{range}\PY{p}{(}\PY{n}{data}\PY{o}{.}\PY{n}{shape}\PY{p}{[}\PY{l+m+mi}{0}\PY{p}{]}\PY{p}{)}\PY{p}{:}
                 \PY{k}{if} \PY{o+ow}{not} \PY{n}{np}\PY{o}{.}\PY{n}{isfinite}\PY{p}{(}\PY{n}{data}\PY{p}{[}\PY{n}{i}\PY{p}{]}\PY{p}{)}\PY{p}{:}
                     \PY{n}{data}\PY{p}{[}\PY{n}{i}\PY{p}{]} \PY{o}{=} \PY{n}{data}\PY{p}{[}\PY{n}{i}\PY{o}{\PYZhy{}}\PY{l+m+mi}{1}\PY{p}{]}
             \PY{k}{return} \PY{n}{data}

         \PY{n}{year} \PY{o}{=} \PY{n}{remove\PYZus{}nans}\PY{p}{(}\PY{n}{year}\PY{p}{)}
         \PY{n}{year\PYZus{}linear\PYZus{}det} \PY{o}{=} \PY{n}{signal}\PY{o}{.}\PY{n}{detrend}\PY{p}{(}\PY{n}{year}\PY{p}{,} \PY{n}{axis}\PY{o}{=}\PY{l+m+mi}{0}\PY{p}{,} \PY{n+nb}{type}\PY{o}{=}\PY{l+s}{\PYZsq{}}\PY{l+s}{linear}\PY{l+s}{\PYZsq{}}\PY{p}{)}
\end{Verbatim}

    Single Year Linear Detrend Time Series

    \begin{Verbatim}[commandchars=\\\{\}]
{\color{incolor}In [{\color{incolor}14}]:} \PY{n}{plt}\PY{o}{.}\PY{n}{figure}\PY{p}{(}\PY{n}{figsize}\PY{o}{=}\PY{p}{(}\PY{l+m+mi}{15}\PY{p}{,} \PY{l+m+mi}{5}\PY{p}{)}\PY{p}{)}
         \PY{n}{plt}\PY{o}{.}\PY{n}{plot}\PY{p}{(}\PY{n}{year\PYZus{}linear\PYZus{}det}\PY{p}{)}
         \PY{n}{plt}\PY{o}{.}\PY{n}{title}\PY{p}{(}\PY{l+s}{\PYZdq{}}\PY{l+s}{Linear Detrend Single Year}\PY{l+s}{\PYZdq{}}\PY{p}{)}
\end{Verbatim}

            \begin{Verbatim}[commandchars=\\\{\}]
{\color{outcolor}Out[{\color{outcolor}14}]:} <matplotlib.text.Text at 0x7fd8f0b6b490>
\end{Verbatim}

    \begin{center}
    \adjustimage{max size={0.9\linewidth}{0.9\paperheight}}{analysis_files/analysis_26_1.png}
    \end{center}
    { \hspace*{\fill} \\}

    \subsection{Summaries All Data}\label{summaries-all-data}

    Group by Month: Global monthly trend averaging over the 5 samples of
each month

    \begin{Verbatim}[commandchars=\\\{\}]
{\color{incolor}In [{\color{incolor}16}]:} \PY{n}{grp} \PY{o}{=} \PY{n}{df}\PY{o}{.}\PY{n}{groupby}\PY{p}{(}\PY{n}{df}\PY{o}{.}\PY{n}{date}\PY{o}{.}\PY{n}{map}\PY{p}{(}\PY{k}{lambda} \PY{n}{x}\PY{p}{:}\PY{n}{x}\PY{o}{.}\PY{n}{month}\PY{p}{)}\PY{p}{)}
         \PY{n}{grp}\PY{o}{.}\PY{n}{mean}\PY{p}{(}\PY{p}{)}\PY{p}{[}\PY{l+s}{\PYZsq{}}\PY{l+s}{car.count}\PY{l+s}{\PYZsq{}}\PY{p}{]}\PY{o}{.}\PY{n}{plot}\PY{p}{(}\PY{n}{figsize}\PY{o}{=}\PY{p}{(}\PY{l+m+mi}{15}\PY{p}{,}\PY{l+m+mi}{5}\PY{p}{)}\PY{p}{,} \PY{n}{title}\PY{o}{=}\PY{l+s}{\PYZdq{}}\PY{l+s}{Group by Month}\PY{l+s}{\PYZdq{}}\PY{p}{)}
\end{Verbatim}

            \begin{Verbatim}[commandchars=\\\{\}]
{\color{outcolor}Out[{\color{outcolor}16}]:} <matplotlib.axes.\_subplots.AxesSubplot at 0x7fd8f0ad38d0>
\end{Verbatim}

    \begin{center}
    \adjustimage{max size={0.9\linewidth}{0.9\paperheight}}{analysis_files/analysis_29_1.png}
    \end{center}
    { \hspace*{\fill} \\}

    Group by Year: Global year trend averaging over the 5 samples of each
year

    \begin{Verbatim}[commandchars=\\\{\}]
{\color{incolor}In [{\color{incolor}17}]:} \PY{n}{grp} \PY{o}{=} \PY{n}{df}\PY{o}{.}\PY{n}{groupby}\PY{p}{(}\PY{n}{df}\PY{o}{.}\PY{n}{date}\PY{o}{.}\PY{n}{map}\PY{p}{(}\PY{k}{lambda} \PY{n}{x}\PY{p}{:}\PY{n}{x}\PY{o}{.}\PY{n}{year}\PY{p}{)}\PY{p}{)}
         \PY{n}{grp}\PY{o}{.}\PY{n}{mean}\PY{p}{(}\PY{p}{)}\PY{p}{[}\PY{l+s}{\PYZsq{}}\PY{l+s}{car.count}\PY{l+s}{\PYZsq{}}\PY{p}{]}\PY{o}{.}\PY{n}{plot}\PY{p}{(}\PY{n}{figsize}\PY{o}{=}\PY{p}{(}\PY{l+m+mi}{15}\PY{p}{,}\PY{l+m+mi}{5}\PY{p}{)}\PY{p}{,} \PY{n}{title}\PY{o}{=}\PY{l+s}{\PYZdq{}}\PY{l+s}{Grouby Year}\PY{l+s}{\PYZdq{}}\PY{p}{)}
\end{Verbatim}

            \begin{Verbatim}[commandchars=\\\{\}]
{\color{outcolor}Out[{\color{outcolor}17}]:} <matplotlib.axes.\_subplots.AxesSubplot at 0x7fd8f0a5fbd0>
\end{Verbatim}

    \begin{center}
    \adjustimage{max size={0.9\linewidth}{0.9\paperheight}}{analysis_files/analysis_31_1.png}
    \end{center}
    { \hspace*{\fill} \\}

    Group by Day: Global day trend averaging over the 5 samples of each day

    \begin{Verbatim}[commandchars=\\\{\}]
{\color{incolor}In [{\color{incolor}18}]:} \PY{n}{grp} \PY{o}{=} \PY{n}{df}\PY{o}{.}\PY{n}{groupby}\PY{p}{(}\PY{n}{df}\PY{o}{.}\PY{n}{date}\PY{o}{.}\PY{n}{map}\PY{p}{(}\PY{k}{lambda} \PY{n}{x}\PY{p}{:}\PY{n}{x}\PY{o}{.}\PY{n}{day}\PY{p}{)}\PY{p}{)}
         \PY{n}{grp}\PY{o}{.}\PY{n}{mean}\PY{p}{(}\PY{p}{)}\PY{p}{[}\PY{l+s}{\PYZsq{}}\PY{l+s}{car.count}\PY{l+s}{\PYZsq{}}\PY{p}{]}\PY{o}{.}\PY{n}{plot}\PY{p}{(}\PY{n}{figsize}\PY{o}{=}\PY{p}{(}\PY{l+m+mi}{15}\PY{p}{,}\PY{l+m+mi}{5}\PY{p}{)}\PY{p}{,} \PY{n}{title}\PY{o}{=}\PY{l+s}{\PYZdq{}}\PY{l+s}{Groupby Day}\PY{l+s}{\PYZdq{}}\PY{p}{)}
\end{Verbatim}

            \begin{Verbatim}[commandchars=\\\{\}]
{\color{outcolor}Out[{\color{outcolor}18}]:} <matplotlib.axes.\_subplots.AxesSubplot at 0x7fd8f842b690>
\end{Verbatim}

    \begin{center}
    \adjustimage{max size={0.9\linewidth}{0.9\paperheight}}{analysis_files/analysis_33_1.png}
    \end{center}
    { \hspace*{\fill} \\}

    \subsection{FFT}\label{fft}

I keep this method handy, I didn't re-write just cut and paste rather
than import so that you can see what's inside.

    \begin{Verbatim}[commandchars=\\\{\}]
{\color{incolor}In [{\color{incolor}21}]:} \PY{k}{def} \PY{n+nf}{fft}\PY{p}{(}\PY{n}{data}\PY{p}{)}\PY{p}{:}
                 \PY{l+s+sd}{\PYZsq{}\PYZsq{}\PYZsq{}Plot FFT using Welch\PYZsq{}s method, daily resolution \PYZsq{}\PYZsq{}\PYZsq{}}
                 \PY{c}{\PYZsh{}plt.figure(figsize=(13, 7))}
                 \PY{n}{f}\PY{p}{,} \PY{n}{y} \PY{o}{=} \PY{n}{signal}\PY{o}{.}\PY{n}{welch}\PY{p}{(}\PY{n}{data}\PY{p}{,} \PY{n}{fs}\PY{o}{=}\PY{l+m+mf}{1.0}\PY{p}{,} \PY{n}{nperseg}\PY{o}{=}\PY{l+m+mi}{256}\PY{p}{,} \PY{n}{noverlap}\PY{o}{=}\PY{l+m+mi}{128}\PY{p}{,} \PY{n}{nfft}\PY{o}{=}\PY{l+m+mi}{512}\PY{p}{,} \PY{n}{scaling}\PY{o}{=}\PY{l+s}{\PYZsq{}}\PY{l+s}{spectrum}\PY{l+s}{\PYZsq{}}\PY{p}{,} \PY{n}{detrend}\PY{o}{=}\PY{l+s}{\PYZdq{}}\PY{l+s}{linear}\PY{l+s}{\PYZdq{}}\PY{p}{)}

                 \PY{n}{interval} \PY{o}{=} \PY{l+m+mi}{3} \PY{c}{\PYZsh{} days}
                 \PY{n}{periods} \PY{o}{=} \PY{n}{np}\PY{o}{.}\PY{n}{round}\PY{p}{(}\PY{l+m+mi}{1}\PY{o}{/}\PY{n}{f}\PY{p}{[}\PY{l+m+mi}{0}\PY{p}{:}\PY{p}{:}\PY{n}{interval}\PY{p}{]}\PY{p}{,} \PY{l+m+mi}{1}\PY{p}{)}
                 \PY{c}{\PYZsh{} clean up frequency of 0 Hz}
                 \PY{n}{periods}\PY{p}{[}\PY{l+m+mi}{0}\PY{p}{]} \PY{o}{=} \PY{l+m+mi}{0}

                 \PY{n}{frqs} \PY{o}{=} \PY{n}{f}\PY{p}{[}\PY{l+m+mi}{0}\PY{p}{:}\PY{p}{:}\PY{n}{interval}\PY{p}{]}
                 \PY{n}{plt}\PY{o}{.}\PY{n}{xticks}\PY{p}{(}\PY{n}{frqs}\PY{p}{,} \PY{n}{periods}\PY{p}{,} \PY{n}{rotation}\PY{o}{=}\PY{l+s}{\PYZdq{}}\PY{l+s}{vertical}\PY{l+s}{\PYZdq{}}\PY{p}{)}

                 \PY{n}{plt}\PY{o}{.}\PY{n}{plot}\PY{p}{(}\PY{n}{f}\PY{p}{,} \PY{n}{y}\PY{p}{)}

                 \PY{n}{plt}\PY{o}{.}\PY{n}{grid}\PY{p}{(}\PY{n+nb+bp}{True}\PY{p}{)} \PY{c}{\PYZsh{} not working likely b/c of conflict with seaborn artist}
                 \PY{n}{plt}\PY{o}{.}\PY{n}{title}\PY{p}{(}\PY{l+s}{\PYZdq{}}\PY{l+s}{Welch FFT: Counts}\PY{l+s}{\PYZdq{}}\PY{p}{)}
                 \PY{n}{plt}\PY{o}{.}\PY{n}{ylabel}\PY{p}{(}\PY{l+s}{\PYZdq{}}\PY{l+s}{Relative ratio of spectrum}\PY{l+s}{\PYZdq{}}\PY{p}{)}
                 \PY{n}{plt}\PY{o}{.}\PY{n}{xlabel}\PY{p}{(}\PY{l+s}{\PYZdq{}}\PY{l+s}{Number of days in a period}\PY{l+s}{\PYZdq{}}\PY{p}{)}

                 \PY{k}{return} \PY{n}{f}\PY{p}{,} \PY{n}{y}\PY{p}{,} \PY{n}{frqs}
\end{Verbatim}

    \begin{Verbatim}[commandchars=\\\{\}]
{\color{incolor}In [{\color{incolor}22}]:} \PY{n}{plt}\PY{o}{.}\PY{n}{figure}\PY{p}{(}\PY{n}{figsize}\PY{o}{=}\PY{p}{(}\PY{l+m+mi}{15}\PY{p}{,}\PY{l+m+mi}{5}\PY{p}{)}\PY{p}{)}
         \PY{n}{counts} \PY{o}{=} \PY{n}{remove\PYZus{}nans}\PY{p}{(}\PY{n}{df}\PY{p}{[}\PY{l+s}{\PYZsq{}}\PY{l+s}{car.count}\PY{l+s}{\PYZsq{}}\PY{p}{]}\PY{p}{)}
         \PY{n}{f}\PY{p}{,} \PY{n}{y}\PY{p}{,} \PY{n}{frqs} \PY{o}{=} \PY{n}{fft}\PY{p}{(}\PY{n}{counts}\PY{p}{)}
\end{Verbatim}

    \begin{center}
    \adjustimage{max size={0.9\linewidth}{0.9\paperheight}}{analysis_files/analysis_36_0.png}
    \end{center}
    { \hspace*{\fill} \\}

    \begin{Verbatim}[commandchars=\\\{\}]
{\color{incolor}In [{\color{incolor}248}]:} \PY{o}{\PYZpc{}\PYZpc{}}\PY{k}{bash}
          date
\end{Verbatim}

    \begin{Verbatim}[commandchars=\\\{\}]
Mon Mar 14 16:29:43 PDT 2016
    \end{Verbatim}

    \section{Summary Write Up}\label{summary-write-up}

    Longer time scale: If we were asked about global count increase, we
could use the yearly and fit a quadratic to see that global trend.

Shorter time scale: There's probably a strong correlation with some
external related time series data that characterizes typical driving use
cases.

Clearly there's an 85 day period with freqency $f = 1/T$ giving
$f \approx$ 3 month recurring feature of high counts.

    \section{Future Analysis}\label{future-analysis}

    I'd like to fit a quadratic to the global time series and report that
fit for prediction or detrending purposes. The Welch FFT shouldn't
improve though, because of the sub sampling and linear detrend. The 2nd
order curve is rather flat and should look linear in most subsamples.

The Welch FFT I pulled out was not optimized for this data set any known
confounds so I'd need to sanity check it more and tweak some values.

    \section{Note About Time}\label{note-about-time}

The additional 30 min is mostly from me trying to be clever with
vectorizing and reminding myself how I like to use datetime objects in
Pandas.

    \begin{Verbatim}[commandchars=\\\{\}]
{\color{incolor}In [{\color{incolor} }]:}
\end{Verbatim}


    % Add a bibliography block to the postdoc



    \end{document}
