
% Default to the notebook output style




% Inherit from the specified cell style.





\documentclass{article}



    \usepackage{graphicx} % Used to insert images
    \usepackage{adjustbox} % Used to constrain images to a maximum size
    \usepackage{color} % Allow colors to be defined
    \usepackage{enumerate} % Needed for markdown enumerations to work
    \usepackage{geometry} % Used to adjust the document margins
    \usepackage{amsmath} % Equations
    \usepackage{amssymb} % Equations
    \usepackage{eurosym} % defines \euro
    \usepackage[mathletters]{ucs} % Extended unicode (utf-8) support
    \usepackage[utf8x]{inputenc} % Allow utf-8 characters in the tex document
    \usepackage{fancyvrb} % verbatim replacement that allows latex
    \usepackage{grffile} % extends the file name processing of package graphics
                         % to support a larger range
    % The hyperref package gives us a pdf with properly built
    % internal navigation ('pdf bookmarks' for the table of contents,
    % internal cross-reference links, web links for URLs, etc.)
    \usepackage{hyperref}
    \usepackage{longtable} % longtable support required by pandoc >1.10
    \usepackage{booktabs}  % table support for pandoc > 1.12.2
    \usepackage{courier}




    \definecolor{orange}{cmyk}{0,0.4,0.8,0.2}
    \definecolor{darkorange}{rgb}{.71,0.21,0.01}
    \definecolor{darkgreen}{rgb}{.12,.54,.11}
    \definecolor{myteal}{rgb}{.26, .44, .56}
    \definecolor{gray}{gray}{0.45}
    \definecolor{lightgray}{gray}{.95}
    \definecolor{mediumgray}{gray}{.8}
    \definecolor{inputbackground}{rgb}{.95, .95, .85}
    \definecolor{outputbackground}{rgb}{.95, .95, .95}
    \definecolor{traceback}{rgb}{1, .95, .95}
    % ansi colors
    \definecolor{red}{rgb}{.6,0,0}
    \definecolor{green}{rgb}{0,.65,0}
    \definecolor{brown}{rgb}{0.6,0.6,0}
    \definecolor{blue}{rgb}{0,.145,.698}
    \definecolor{purple}{rgb}{.698,.145,.698}
    \definecolor{cyan}{rgb}{0,.698,.698}
    \definecolor{lightgray}{gray}{0.5}

    % bright ansi colors
    \definecolor{darkgray}{gray}{0.25}
    \definecolor{lightred}{rgb}{1.0,0.39,0.28}
    \definecolor{lightgreen}{rgb}{0.48,0.99,0.0}
    \definecolor{lightblue}{rgb}{0.53,0.81,0.92}
    \definecolor{lightpurple}{rgb}{0.87,0.63,0.87}
    \definecolor{lightcyan}{rgb}{0.5,1.0,0.83}

    % commands and environments needed by pandoc snippets
    % extracted from the output of `pandoc -s`
    \DefineVerbatimEnvironment{Highlighting}{Verbatim}{commandchars=\\\{\}}
    % Add ',fontsize=\small' for more characters per line
    \newenvironment{Shaded}{}{}
    \newcommand{\KeywordTok}[1]{\textcolor[rgb]{0.00,0.44,0.13}{\textbf{{#1}}}}
    \newcommand{\DataTypeTok}[1]{\textcolor[rgb]{0.56,0.13,0.00}{{#1}}}
    \newcommand{\DecValTok}[1]{\textcolor[rgb]{0.25,0.63,0.44}{{#1}}}
    \newcommand{\BaseNTok}[1]{\textcolor[rgb]{0.25,0.63,0.44}{{#1}}}
    \newcommand{\FloatTok}[1]{\textcolor[rgb]{0.25,0.63,0.44}{{#1}}}
    \newcommand{\CharTok}[1]{\textcolor[rgb]{0.25,0.44,0.63}{{#1}}}
    \newcommand{\StringTok}[1]{\textcolor[rgb]{0.25,0.44,0.63}{{#1}}}
    \newcommand{\CommentTok}[1]{\textcolor[rgb]{0.38,0.63,0.69}{\textit{{#1}}}}
    \newcommand{\OtherTok}[1]{\textcolor[rgb]{0.00,0.44,0.13}{{#1}}}
    \newcommand{\AlertTok}[1]{\textcolor[rgb]{1.00,0.00,0.00}{\textbf{{#1}}}}
    \newcommand{\FunctionTok}[1]{\textcolor[rgb]{0.02,0.16,0.49}{{#1}}}
    \newcommand{\RegionMarkerTok}[1]{{#1}}
    \newcommand{\ErrorTok}[1]{\textcolor[rgb]{1.00,0.00,0.00}{\textbf{{#1}}}}
    \newcommand{\NormalTok}[1]{{#1}}

    % Define a nice break command that doesn't care if a line doesn't already
    % exist.
    \def\br{\hspace*{\fill} \\* }
    % Math Jax compatability definitions
    \def\gt{>}
    \def\lt{<}
    % Document parameters
    \title{Time Series Analysis:Recovering A Stochastic Signal}
    \author{Daniel Cuneo}




    % Pygments definitions

\makeatletter
\def\PY@reset{\let\PY@it=\relax \let\PY@bf=\relax%
    \let\PY@ul=\relax \let\PY@tc=\relax%
    \let\PY@bc=\relax \let\PY@ff=\relax}
\def\PY@tok#1{\csname PY@tok@#1\endcsname}
\def\PY@toks#1+{\ifx\relax#1\empty\else%
    \PY@tok{#1}\expandafter\PY@toks\fi}
\def\PY@do#1{\PY@bc{\PY@tc{\PY@ul{%
    \PY@it{\PY@bf{\PY@ff{#1}}}}}}}
\def\PY#1#2{\PY@reset\PY@toks#1+\relax+\PY@do{#2}}

\expandafter\def\csname PY@tok@gd\endcsname{\def\PY@tc##1{\textcolor[rgb]{0.63,0.00,0.00}{##1}}}
\expandafter\def\csname PY@tok@gu\endcsname{\let\PY@bf=\textbf\def\PY@tc##1{\textcolor[rgb]{0.50,0.00,0.50}{##1}}}
\expandafter\def\csname PY@tok@gt\endcsname{\def\PY@tc##1{\textcolor[rgb]{0.00,0.27,0.87}{##1}}}
\expandafter\def\csname PY@tok@gs\endcsname{\let\PY@bf=\textbf}
\expandafter\def\csname PY@tok@gr\endcsname{\def\PY@tc##1{\textcolor[rgb]{1.00,0.00,0.00}{##1}}}
\expandafter\def\csname PY@tok@cm\endcsname{\let\PY@it=\textit\def\PY@tc##1{\textcolor[rgb]{0.25,0.50,0.50}{##1}}}
\expandafter\def\csname PY@tok@vg\endcsname{\def\PY@tc##1{\textcolor[rgb]{0.10,0.09,0.49}{##1}}}
\expandafter\def\csname PY@tok@m\endcsname{\def\PY@tc##1{\textcolor[rgb]{0.40,0.40,0.40}{##1}}}
\expandafter\def\csname PY@tok@mh\endcsname{\def\PY@tc##1{\textcolor[rgb]{0.40,0.40,0.40}{##1}}}
\expandafter\def\csname PY@tok@go\endcsname{\def\PY@tc##1{\textcolor[rgb]{0.53,0.53,0.53}{##1}}}
\expandafter\def\csname PY@tok@ge\endcsname{\let\PY@it=\textit}
\expandafter\def\csname PY@tok@vc\endcsname{\def\PY@tc##1{\textcolor[rgb]{0.10,0.09,0.49}{##1}}}
\expandafter\def\csname PY@tok@il\endcsname{\def\PY@tc##1{\textcolor[rgb]{0.40,0.40,0.40}{##1}}}
\expandafter\def\csname PY@tok@cs\endcsname{\let\PY@it=\textit\def\PY@tc##1{\textcolor[rgb]{0.25,0.50,0.50}{##1}}}
\expandafter\def\csname PY@tok@cp\endcsname{\def\PY@tc##1{\textcolor[rgb]{0.74,0.48,0.00}{##1}}}
\expandafter\def\csname PY@tok@gi\endcsname{\def\PY@tc##1{\textcolor[rgb]{0.00,0.63,0.00}{##1}}}
\expandafter\def\csname PY@tok@gh\endcsname{\let\PY@bf=\textbf\def\PY@tc##1{\textcolor[rgb]{0.00,0.00,0.50}{##1}}}
\expandafter\def\csname PY@tok@ni\endcsname{\let\PY@bf=\textbf\def\PY@tc##1{\textcolor[rgb]{0.60,0.60,0.60}{##1}}}
\expandafter\def\csname PY@tok@nl\endcsname{\def\PY@tc##1{\textcolor[rgb]{0.63,0.63,0.00}{##1}}}
\expandafter\def\csname PY@tok@nn\endcsname{\let\PY@bf=\textbf\def\PY@tc##1{\textcolor[rgb]{0.00,0.00,1.00}{##1}}}
\expandafter\def\csname PY@tok@no\endcsname{\def\PY@tc##1{\textcolor[rgb]{0.53,0.00,0.00}{##1}}}
\expandafter\def\csname PY@tok@na\endcsname{\def\PY@tc##1{\textcolor[rgb]{0.49,0.56,0.16}{##1}}}
\expandafter\def\csname PY@tok@nb\endcsname{\def\PY@tc##1{\textcolor[rgb]{0.00,0.50,0.00}{##1}}}
\expandafter\def\csname PY@tok@nc\endcsname{\let\PY@bf=\textbf\def\PY@tc##1{\textcolor[rgb]{0.00,0.00,1.00}{##1}}}
\expandafter\def\csname PY@tok@nd\endcsname{\def\PY@tc##1{\textcolor[rgb]{0.67,0.13,1.00}{##1}}}
\expandafter\def\csname PY@tok@ne\endcsname{\let\PY@bf=\textbf\def\PY@tc##1{\textcolor[rgb]{0.82,0.25,0.23}{##1}}}
\expandafter\def\csname PY@tok@nf\endcsname{\def\PY@tc##1{\textcolor[rgb]{0.00,0.00,1.00}{##1}}}
\expandafter\def\csname PY@tok@si\endcsname{\let\PY@bf=\textbf\def\PY@tc##1{\textcolor[rgb]{0.73,0.40,0.53}{##1}}}
\expandafter\def\csname PY@tok@s2\endcsname{\def\PY@tc##1{\textcolor[rgb]{0.73,0.13,0.13}{##1}}}
\expandafter\def\csname PY@tok@vi\endcsname{\def\PY@tc##1{\textcolor[rgb]{0.10,0.09,0.49}{##1}}}
\expandafter\def\csname PY@tok@nt\endcsname{\let\PY@bf=\textbf\def\PY@tc##1{\textcolor[rgb]{0.00,0.50,0.00}{##1}}}
\expandafter\def\csname PY@tok@nv\endcsname{\def\PY@tc##1{\textcolor[rgb]{0.10,0.09,0.49}{##1}}}
\expandafter\def\csname PY@tok@s1\endcsname{\def\PY@tc##1{\textcolor[rgb]{0.73,0.13,0.13}{##1}}}
\expandafter\def\csname PY@tok@kd\endcsname{\let\PY@bf=\textbf\def\PY@tc##1{\textcolor[rgb]{0.00,0.50,0.00}{##1}}}
\expandafter\def\csname PY@tok@sh\endcsname{\def\PY@tc##1{\textcolor[rgb]{0.73,0.13,0.13}{##1}}}
\expandafter\def\csname PY@tok@sc\endcsname{\def\PY@tc##1{\textcolor[rgb]{0.73,0.13,0.13}{##1}}}
\expandafter\def\csname PY@tok@sx\endcsname{\def\PY@tc##1{\textcolor[rgb]{0.00,0.50,0.00}{##1}}}
\expandafter\def\csname PY@tok@bp\endcsname{\def\PY@tc##1{\textcolor[rgb]{0.00,0.50,0.00}{##1}}}
\expandafter\def\csname PY@tok@c1\endcsname{\let\PY@it=\textit\def\PY@tc##1{\textcolor[rgb]{0.25,0.50,0.50}{##1}}}
\expandafter\def\csname PY@tok@kc\endcsname{\let\PY@bf=\textbf\def\PY@tc##1{\textcolor[rgb]{0.00,0.50,0.00}{##1}}}
\expandafter\def\csname PY@tok@c\endcsname{\let\PY@it=\textit\def\PY@tc##1{\textcolor[rgb]{0.25,0.50,0.50}{##1}}}
\expandafter\def\csname PY@tok@mf\endcsname{\def\PY@tc##1{\textcolor[rgb]{0.40,0.40,0.40}{##1}}}
\expandafter\def\csname PY@tok@err\endcsname{\def\PY@bc##1{\setlength{\fboxsep}{0pt}\fcolorbox[rgb]{1.00,0.00,0.00}{1,1,1}{\strut ##1}}}
\expandafter\def\csname PY@tok@mb\endcsname{\def\PY@tc##1{\textcolor[rgb]{0.40,0.40,0.40}{##1}}}
\expandafter\def\csname PY@tok@ss\endcsname{\def\PY@tc##1{\textcolor[rgb]{0.10,0.09,0.49}{##1}}}
\expandafter\def\csname PY@tok@sr\endcsname{\def\PY@tc##1{\textcolor[rgb]{0.73,0.40,0.53}{##1}}}
\expandafter\def\csname PY@tok@mo\endcsname{\def\PY@tc##1{\textcolor[rgb]{0.40,0.40,0.40}{##1}}}
\expandafter\def\csname PY@tok@kn\endcsname{\let\PY@bf=\textbf\def\PY@tc##1{\textcolor[rgb]{0.00,0.50,0.00}{##1}}}
\expandafter\def\csname PY@tok@mi\endcsname{\def\PY@tc##1{\textcolor[rgb]{0.40,0.40,0.40}{##1}}}
\expandafter\def\csname PY@tok@gp\endcsname{\let\PY@bf=\textbf\def\PY@tc##1{\textcolor[rgb]{0.00,0.00,0.50}{##1}}}
\expandafter\def\csname PY@tok@o\endcsname{\def\PY@tc##1{\textcolor[rgb]{0.40,0.40,0.40}{##1}}}
\expandafter\def\csname PY@tok@kr\endcsname{\let\PY@bf=\textbf\def\PY@tc##1{\textcolor[rgb]{0.00,0.50,0.00}{##1}}}
\expandafter\def\csname PY@tok@s\endcsname{\def\PY@tc##1{\textcolor[rgb]{0.73,0.13,0.13}{##1}}}
\expandafter\def\csname PY@tok@kp\endcsname{\def\PY@tc##1{\textcolor[rgb]{0.00,0.50,0.00}{##1}}}
\expandafter\def\csname PY@tok@w\endcsname{\def\PY@tc##1{\textcolor[rgb]{0.73,0.73,0.73}{##1}}}
\expandafter\def\csname PY@tok@kt\endcsname{\def\PY@tc##1{\textcolor[rgb]{0.69,0.00,0.25}{##1}}}
\expandafter\def\csname PY@tok@ow\endcsname{\let\PY@bf=\textbf\def\PY@tc##1{\textcolor[rgb]{0.67,0.13,1.00}{##1}}}
\expandafter\def\csname PY@tok@sb\endcsname{\def\PY@tc##1{\textcolor[rgb]{0.73,0.13,0.13}{##1}}}
\expandafter\def\csname PY@tok@k\endcsname{\let\PY@bf=\textbf\def\PY@tc##1{\textcolor[rgb]{0.00,0.50,0.00}{##1}}}
\expandafter\def\csname PY@tok@se\endcsname{\let\PY@bf=\textbf\def\PY@tc##1{\textcolor[rgb]{0.73,0.40,0.13}{##1}}}
\expandafter\def\csname PY@tok@sd\endcsname{\let\PY@it=\textit\def\PY@tc##1{\textcolor[rgb]{0.73,0.13,0.13}{##1}}}

\def\PYZbs{\char`\\}
\def\PYZus{\char`\_}
\def\PYZob{\char`\{}
\def\PYZcb{\char`\}}
\def\PYZca{\char`\^}
\def\PYZam{\char`\&}
\def\PYZlt{\char`\<}
\def\PYZgt{\char`\>}
\def\PYZsh{\char`\#}
\def\PYZpc{\char`\%}
\def\PYZdl{\char`\$}
\def\PYZhy{\char`\-}
\def\PYZsq{\char`\'}
\def\PYZdq{\char`\"}
\def\PYZti{\char`\~}
% for compatibility with earlier versions
\def\PYZat{@}
\def\PYZlb{[}
\def\PYZrb{]}
\makeatother


    % Exact colors from NB
    \definecolor{incolor}{rgb}{0.0, 0.0, 0.5}
    \definecolor{outcolor}{rgb}{0.545, 0.0, 0.0}




    % Prevent overflowing lines due to hard-to-break entities
    \sloppy
    % Setup hyperref package
    \hypersetup{
      breaklinks=true,  % so long urls are correctly broken across lines
      colorlinks=true,
      urlcolor=blue,
      linkcolor=darkorange,
      citecolor=darkgreen,
      }
    % Slightly bigger margins than the latex defaults

    \geometry{verbose,tmargin=1in,bmargin=1in,lmargin=1in,rmargin=1in}



    \begin{document}


    \maketitle




    \section{Time Series Analysis}\label{time-series-analysis}

\subsection{Recovering A Stochastic
Signal}\label{recovering-a-stochastic-signal}

\begin{itemize}
\itemsep1pt\parskip0pt\parsep0pt
\item
  By Daniel Cuneo
\end{itemize}

This is a pretty basic example of how to filter and recover a random
signal from a time series that that has a linear combination of
confounding noise.

    \begin{Verbatim}[commandchars=\\\{\}]
{\color{incolor}In [{\color{incolor}1}]:} \PY{c}{\PYZsh{}\PYZpc{}install\PYZus{}ext https://raw.githubusercontent.com/rasbt/watermark/master/watermark.py}
        \PY{o}{\PYZpc{}}\PY{k}{reload\PYZus{}ext} watermark
        \PY{o}{\PYZpc{}}\PY{k}{watermark} \PYZhy{}p numpy,scipy,pandas,matplotlib
\end{Verbatim}

    \begin{Verbatim}[commandchars=\\\{\}]
numpy 1.10.1
scipy 0.16.0
pandas 0.16.2
matplotlib 1.4.0
    \end{Verbatim}

    \begin{Verbatim}[commandchars=\\\{\}]
{\color{incolor}In [{\color{incolor}2}]:} \PY{k+kn}{import} \PY{n+nn}{matplotlib.pyplot} \PY{k+kn}{as} \PY{n+nn}{plt}
        \PY{k+kn}{import} \PY{n+nn}{numpy} \PY{k+kn}{as} \PY{n+nn}{np}
        \PY{k+kn}{import} \PY{n+nn}{pandas} \PY{k+kn}{as} \PY{n+nn}{pd}
        \PY{k+kn}{import} \PY{n+nn}{scipy.signal} \PY{k+kn}{as} \PY{n+nn}{signal}
        \PY{o}{\PYZpc{}}\PY{k}{matplotlib} inline
\end{Verbatim}

    \begin{Verbatim}[commandchars=\\\{\}]
{\color{incolor}In [{\color{incolor}3}]:} \PY{n}{df} \PY{o}{=} \PY{n}{pd}\PY{o}{.}\PY{n}{read\PYZus{}csv}\PY{p}{(}\PY{l+s}{\PYZdq{}}\PY{l+s}{/home/daniel/git/Python2.7/DataScience/notebooks/TimeSeries/data.csv}\PY{l+s}{\PYZdq{}}\PY{p}{)}
\end{Verbatim}

    \begin{Verbatim}[commandchars=\\\{\}]
{\color{incolor}In [{\color{incolor}4}]:} \PY{n}{df}\PY{o}{.}\PY{n}{head}\PY{p}{(}\PY{p}{)}
\end{Verbatim}

            \begin{Verbatim}[commandchars=\\\{\}]
{\color{outcolor}Out[{\color{outcolor}4}]:}          date day.of.week  car.count  weather
        0  2010-01-01      friday       94.5     -0.1
        1  2010-01-02    saturday      108.4     -2.4
        2  2010-01-03      sunday      105.5     -0.5
        3  2010-01-04      monday      109.6     -2.1
        4  2010-01-05     tuesday      116.1      1.9
\end{Verbatim}

    \begin{Verbatim}[commandchars=\\\{\}]
{\color{incolor}In [{\color{incolor}5}]:} \PY{c}{\PYZsh{} I like using Pandas b/c of the datetime features, resample or groupby}
        \PY{n}{df}\PY{p}{[}\PY{l+s}{\PYZsq{}}\PY{l+s}{date}\PY{l+s}{\PYZsq{}}\PY{p}{]} \PY{o}{=} \PY{n}{pd}\PY{o}{.}\PY{n}{to\PYZus{}datetime}\PY{p}{(}\PY{n}{df}\PY{p}{[}\PY{l+s}{\PYZsq{}}\PY{l+s}{date}\PY{l+s}{\PYZsq{}}\PY{p}{]}\PY{p}{)}
        \PY{n}{df}\PY{o}{.}\PY{n}{set\PYZus{}index}\PY{p}{(}\PY{n}{df}\PY{p}{[}\PY{l+s}{\PYZsq{}}\PY{l+s}{date}\PY{l+s}{\PYZsq{}}\PY{p}{]}\PY{p}{,} \PY{n}{inplace}\PY{o}{=}\PY{n+nb+bp}{True}\PY{p}{)}
\end{Verbatim}

    \subsection{Initial Plot}\label{initial-plot}

    \begin{Verbatim}[commandchars=\\\{\}]
{\color{incolor}In [{\color{incolor}6}]:} \PY{n}{df}\PY{p}{[}\PY{l+s}{\PYZsq{}}\PY{l+s}{car.count}\PY{l+s}{\PYZsq{}}\PY{p}{]}\PY{o}{.}\PY{n}{plot}\PY{p}{(}\PY{n}{rot}\PY{o}{=}\PY{l+m+mi}{90}\PY{p}{,} \PY{n}{figsize}\PY{o}{=}\PY{p}{(}\PY{l+m+mi}{25}\PY{p}{,} \PY{l+m+mi}{5}\PY{p}{)}\PY{p}{)}\PY{p}{;}
\end{Verbatim}

    \begin{center}
    \adjustimage{max size={0.9\linewidth}{0.9\paperheight}}{time_series_files/time_series_7_0.png}
    \end{center}
    { \hspace*{\fill} \\}

    I'd guess that we have a linear combination of a quadratic, sinusoid and
random stochastic signal.

    \subsubsection{Single Year Analysis: first year in the
record}\label{single-year-analysis-first-year-in-the-record}

    \begin{Verbatim}[commandchars=\\\{\}]
{\color{incolor}In [{\color{incolor}7}]:} \PY{c}{\PYZsh{} year = df[\PYZsq{}car.count\PYZsq{}][0:365]  if you are in a rush}
        \PY{n}{year} \PY{o}{=} \PY{n}{df}\PY{p}{[}\PY{n}{df}\PY{p}{[}\PY{l+s}{\PYZsq{}}\PY{l+s}{date}\PY{l+s}{\PYZsq{}}\PY{p}{]} \PY{o}{\PYZlt{}} \PY{n}{pd}\PY{o}{.}\PY{n}{to\PYZus{}datetime}\PY{p}{(}\PY{l+s}{\PYZsq{}}\PY{l+s}{20110101}\PY{l+s}{\PYZsq{}}\PY{p}{)}\PY{p}{]}\PY{p}{[}\PY{l+s}{\PYZsq{}}\PY{l+s}{car.count}\PY{l+s}{\PYZsq{}}\PY{p}{]}
        \PY{n}{year}\PY{o}{.}\PY{n}{plot}\PY{p}{(}\PY{n}{rot}\PY{o}{=}\PY{l+m+mi}{90}\PY{p}{,} \PY{n}{figsize}\PY{o}{=}\PY{p}{(}\PY{l+m+mi}{25}\PY{p}{,} \PY{l+m+mi}{7}\PY{p}{)}\PY{p}{,} \PY{n}{title}\PY{o}{=}\PY{l+s}{\PYZdq{}}\PY{l+s}{Single Year Analysis: First Year on Record}\PY{l+s}{\PYZdq{}}\PY{p}{)}\PY{p}{;}
\end{Verbatim}

    \begin{center}
    \adjustimage{max size={0.9\linewidth}{0.9\paperheight}}{time_series_files/time_series_10_0.png}
    \end{center}
    { \hspace*{\fill} \\}

    It's not easy to see, but there are missing values in the series. We
need to treat those.

    \begin{Verbatim}[commandchars=\\\{\}]
{\color{incolor}In [{\color{incolor}8}]:} \PY{c}{\PYZsh{}TODO: add to signal processing module}

        \PY{k}{def} \PY{n+nf}{remove\PYZus{}nans}\PY{p}{(}\PY{n}{data}\PY{p}{,} \PY{n}{return\PYZus{}nan\PYZus{}index}\PY{o}{=}\PY{n+nb+bp}{False}\PY{p}{)}\PY{p}{:}
            \PY{n}{nan\PYZus{}ind} \PY{o}{=} \PY{n}{np}\PY{o}{.}\PY{n}{nonzero}\PY{p}{(}\PY{o}{\PYZti{}}\PY{n}{np}\PY{o}{.}\PY{n}{isfinite}\PY{p}{(}\PY{n}{data}\PY{p}{)}\PY{p}{)}\PY{p}{[}\PY{l+m+mi}{0}\PY{p}{]}
            \PY{n}{good\PYZus{}data\PYZus{}ind} \PY{o}{=} \PY{n}{np}\PY{o}{.}\PY{n}{nonzero}\PY{p}{(}\PY{n}{np}\PY{o}{.}\PY{n}{isfinite}\PY{p}{(}\PY{n}{data}\PY{p}{)}\PY{p}{)}\PY{p}{[}\PY{l+m+mi}{0}\PY{p}{]}
            \PY{n}{good\PYZus{}data} \PY{o}{=} \PY{n}{data}\PY{p}{[}\PY{n}{good\PYZus{}data\PYZus{}ind}\PY{p}{]}

            \PY{n}{new\PYZus{}points} \PY{o}{=} \PY{n}{np}\PY{o}{.}\PY{n}{interp}\PY{p}{(}\PY{n}{nan\PYZus{}ind}\PY{p}{,} \PY{n}{good\PYZus{}data\PYZus{}ind}\PY{p}{,} \PY{n}{good\PYZus{}data}\PY{p}{)}
            \PY{n}{data}\PY{p}{[}\PY{n}{nan\PYZus{}ind}\PY{p}{]} \PY{o}{=} \PY{n}{new\PYZus{}points}

            \PY{k}{if} \PY{n}{return\PYZus{}nan\PYZus{}index}\PY{p}{:}
                \PY{k}{return} \PY{n}{data}\PY{p}{,} \PY{n}{nan\PYZus{}ind}
            \PY{k}{else}\PY{p}{:}
                \PY{k}{return} \PY{n}{data}
\end{Verbatim}

    \begin{Verbatim}[commandchars=\\\{\}]
{\color{incolor}In [{\color{incolor}9}]:} \PY{n}{year}\PY{p}{,} \PY{n}{nan\PYZus{}ind} \PY{o}{=} \PY{n}{remove\PYZus{}nans}\PY{p}{(}\PY{n}{year}\PY{o}{.}\PY{n}{copy}\PY{p}{(}\PY{p}{)}\PY{p}{,} \PY{n}{return\PYZus{}nan\PYZus{}index}\PY{o}{=}\PY{n+nb+bp}{True}\PY{p}{)}
        \PY{n}{year\PYZus{}linear\PYZus{}det} \PY{o}{=} \PY{n}{signal}\PY{o}{.}\PY{n}{detrend}\PY{p}{(}\PY{n}{year}\PY{p}{,} \PY{n}{axis}\PY{o}{=}\PY{l+m+mi}{0}\PY{p}{,} \PY{n+nb}{type}\PY{o}{=}\PY{l+s}{\PYZsq{}}\PY{l+s}{linear}\PY{l+s}{\PYZsq{}}\PY{p}{)}
\end{Verbatim}

    Single Year Linear Detrend Time Series

    \begin{Verbatim}[commandchars=\\\{\}]
{\color{incolor}In [{\color{incolor}10}]:} \PY{n}{plt}\PY{o}{.}\PY{n}{figure}\PY{p}{(}\PY{n}{figsize}\PY{o}{=}\PY{p}{(}\PY{l+m+mi}{25}\PY{p}{,} \PY{l+m+mi}{5}\PY{p}{)}\PY{p}{)}
         \PY{n}{plt}\PY{o}{.}\PY{n}{xticks}\PY{p}{(}\PY{n}{np}\PY{o}{.}\PY{n}{arange}\PY{p}{(}\PY{n}{year}\PY{o}{.}\PY{n}{shape}\PY{p}{[}\PY{l+m+mi}{0}\PY{p}{]}\PY{p}{)}\PY{p}{[}\PY{l+m+mi}{0}\PY{p}{:}\PY{p}{:}\PY{l+m+mi}{30}\PY{p}{]}\PY{p}{,} \PY{n}{year}\PY{o}{.}\PY{n}{index}\PY{p}{[}\PY{l+m+mi}{0}\PY{p}{:}\PY{p}{:}\PY{l+m+mi}{30}\PY{p}{]}\PY{p}{,} \PY{n}{rotation}\PY{o}{=}\PY{l+m+mi}{90}\PY{p}{)}
         \PY{n}{plt}\PY{o}{.}\PY{n}{plot}\PY{p}{(}\PY{n}{year\PYZus{}linear\PYZus{}det}\PY{p}{,} \PY{n}{alpha}\PY{o}{=}\PY{l+m+mf}{0.5}\PY{p}{)}
         \PY{n}{plt}\PY{o}{.}\PY{n}{title}\PY{p}{(}\PY{l+s}{\PYZdq{}}\PY{l+s}{Linear Detrend Single Year}\PY{l+s}{\PYZdq{}}\PY{p}{)}
         \PY{n}{plt}\PY{o}{.}\PY{n}{plot}\PY{p}{(}\PY{n}{nan\PYZus{}ind}\PY{p}{,} \PY{n}{year\PYZus{}linear\PYZus{}det}\PY{p}{[}\PY{n}{nan\PYZus{}ind}\PY{p}{]}\PY{p}{,} \PY{l+s}{\PYZsq{}}\PY{l+s}{rx}\PY{l+s}{\PYZsq{}}\PY{p}{)}
         \PY{n}{plt}\PY{o}{.}\PY{n}{xlim}\PY{p}{(}\PY{l+m+mi}{0}\PY{p}{,} \PY{l+m+mi}{366}\PY{p}{)}
\end{Verbatim}

            \begin{Verbatim}[commandchars=\\\{\}]
{\color{outcolor}Out[{\color{outcolor}10}]:} (0, 366)
\end{Verbatim}

    \begin{center}
    \adjustimage{max size={0.9\linewidth}{0.9\paperheight}}{time_series_files/time_series_15_1.png}
    \end{center}
    { \hspace*{\fill} \\}

    We see a $\approx$ 90 day period here.

Without more insight about the data we don't know if this is a nuisance
or a feature we are looking for.

The NaN replacements look reasonable.

    \subsection{Group By for Basic
Analysis}\label{group-by-for-basic-analysis}

Grouping the data points into bins and taking the mean, is very simialr
to a Fourier Transform.

Pandas makes this easy and there's no reason not to. Especially if the
data is related to business trends.

    \subsubsection{Group by Month: Global monthly trend averaging over the 5
samples of each
month}\label{group-by-month-global-monthly-trend-averaging-over-the-5-samples-of-each-month}

    \begin{Verbatim}[commandchars=\\\{\}]
{\color{incolor}In [{\color{incolor}11}]:} \PY{n}{grp} \PY{o}{=} \PY{n}{df}\PY{o}{.}\PY{n}{groupby}\PY{p}{(}\PY{n}{df}\PY{o}{.}\PY{n}{date}\PY{o}{.}\PY{n}{map}\PY{p}{(}\PY{k}{lambda} \PY{n}{x}\PY{p}{:}\PY{n}{x}\PY{o}{.}\PY{n}{month}\PY{p}{)}\PY{p}{)}
         \PY{n}{grp}\PY{o}{.}\PY{n}{mean}\PY{p}{(}\PY{p}{)}\PY{p}{[}\PY{l+s}{\PYZsq{}}\PY{l+s}{car.count}\PY{l+s}{\PYZsq{}}\PY{p}{]}\PY{o}{.}\PY{n}{plot}\PY{p}{(}\PY{n}{figsize}\PY{o}{=}\PY{p}{(}\PY{l+m+mi}{15}\PY{p}{,}\PY{l+m+mi}{5}\PY{p}{)}\PY{p}{,} \PY{n}{title}\PY{o}{=}\PY{l+s}{\PYZdq{}}\PY{l+s}{Group by Month}\PY{l+s}{\PYZdq{}}\PY{p}{)}\PY{p}{;}
\end{Verbatim}

    \begin{center}
    \adjustimage{max size={0.9\linewidth}{0.9\paperheight}}{time_series_files/time_series_19_0.png}
    \end{center}
    { \hspace*{\fill} \\}

    A group-by is sort of like a Fourier Transform where we choose just one
frequency bin. There's the sinusodial period $\approx$ 90 days.

    \subsubsection{Group by Day: Global day trend averaging over the 5
samples of each
day}\label{group-by-day-global-day-trend-averaging-over-the-5-samples-of-each-day}

    \begin{Verbatim}[commandchars=\\\{\}]
{\color{incolor}In [{\color{incolor}12}]:} \PY{n}{grp} \PY{o}{=} \PY{n}{df}\PY{o}{.}\PY{n}{groupby}\PY{p}{(}\PY{n}{df}\PY{o}{.}\PY{n}{date}\PY{o}{.}\PY{n}{map}\PY{p}{(}\PY{k}{lambda} \PY{n}{x}\PY{p}{:}\PY{n}{x}\PY{o}{.}\PY{n}{day}\PY{p}{)}\PY{p}{)}
         \PY{n}{grp}\PY{o}{.}\PY{n}{mean}\PY{p}{(}\PY{p}{)}\PY{p}{[}\PY{l+s}{\PYZsq{}}\PY{l+s}{car.count}\PY{l+s}{\PYZsq{}}\PY{p}{]}\PY{o}{.}\PY{n}{plot}\PY{p}{(}\PY{n}{figsize}\PY{o}{=}\PY{p}{(}\PY{l+m+mi}{15}\PY{p}{,}\PY{l+m+mi}{5}\PY{p}{)}\PY{p}{,} \PY{n}{title}\PY{o}{=}\PY{l+s}{\PYZdq{}}\PY{l+s}{Groupby Day}\PY{l+s}{\PYZdq{}}\PY{p}{)}\PY{p}{;}
\end{Verbatim}

    \begin{center}
    \adjustimage{max size={0.9\linewidth}{0.9\paperheight}}{time_series_files/time_series_22_0.png}
    \end{center}
    { \hspace*{\fill} \\}

    It would appear as though the 21st day of each month saw greater count.
However, we should check to see if it is statiscally significant.

    \subsection{Removing Confounds}\label{removing-confounds}

\subsubsection{Quadratic Detrend Using
PolyFit}\label{quadratic-detrend-using-polyfit}

    \begin{Verbatim}[commandchars=\\\{\}]
{\color{incolor}In [{\color{incolor}13}]:} \PY{n}{poly} \PY{o}{=}  \PY{n}{np}\PY{o}{.}\PY{n}{polynomial}\PY{o}{.}\PY{n}{polynomial}
         \PY{n}{counts} \PY{o}{=} \PY{n}{remove\PYZus{}nans}\PY{p}{(}\PY{n}{df}\PY{p}{[}\PY{l+s}{\PYZsq{}}\PY{l+s}{car.count}\PY{l+s}{\PYZsq{}}\PY{p}{]}\PY{o}{.}\PY{n}{copy}\PY{p}{(}\PY{p}{)}\PY{p}{,} \PY{n}{return\PYZus{}nan\PYZus{}index}\PY{o}{=}\PY{n+nb+bp}{False}\PY{p}{)}

         \PY{n}{t} \PY{o}{=} \PY{n}{np}\PY{o}{.}\PY{n}{arange}\PY{p}{(}\PY{n}{df}\PY{o}{.}\PY{n}{shape}\PY{p}{[}\PY{l+m+mi}{0}\PY{p}{]}\PY{p}{)}
         \PY{n}{coefs} \PY{o}{=} \PY{n}{poly}\PY{o}{.}\PY{n}{polyfit}\PY{p}{(}\PY{n}{t}\PY{p}{,} \PY{n}{counts}\PY{p}{,} \PY{n}{deg}\PY{o}{=}\PY{l+m+mi}{2}\PY{p}{,} \PY{n}{full}\PY{o}{=}\PY{n+nb+bp}{False}\PY{p}{)}
         \PY{n}{fit\PYZus{}curve} \PY{o}{=} \PY{n}{poly}\PY{o}{.}\PY{n}{polyval}\PY{p}{(}\PY{n}{t}\PY{p}{,} \PY{n}{coefs}\PY{p}{)}
\end{Verbatim}

    \begin{Verbatim}[commandchars=\\\{\}]
{\color{incolor}In [{\color{incolor}14}]:} \PY{n}{plt}\PY{o}{.}\PY{n}{figure}\PY{p}{(}\PY{n}{figsize}\PY{o}{=}\PY{p}{(}\PY{l+m+mi}{25}\PY{p}{,} \PY{l+m+mi}{5}\PY{p}{)}\PY{p}{)}

         \PY{n}{plt}\PY{o}{.}\PY{n}{plot}\PY{p}{(}\PY{n}{fit\PYZus{}curve}\PY{p}{,} \PY{l+s}{\PYZsq{}}\PY{l+s}{r}\PY{l+s}{\PYZsq{}}\PY{p}{)}
         \PY{n}{plt}\PY{o}{.}\PY{n}{plot}\PY{p}{(}\PY{n}{t}\PY{p}{,} \PY{n}{counts}\PY{p}{,} \PY{n}{alpha}\PY{o}{=}\PY{l+m+mf}{0.5}\PY{p}{)}

         \PY{n}{labels} \PY{o}{=} \PY{n}{df}\PY{p}{[}\PY{l+s}{\PYZsq{}}\PY{l+s}{date}\PY{l+s}{\PYZsq{}}\PY{p}{]}
         \PY{n}{date\PYZus{}str} \PY{o}{=} \PY{n+nb}{map}\PY{p}{(}\PY{k}{lambda} \PY{n}{x}\PY{p}{:} \PY{n+nb}{str}\PY{p}{(}\PY{n}{x}\PY{o}{.}\PY{n}{year}\PY{p}{)} \PY{o}{+} \PY{l+s}{\PYZdq{}}\PY{l+s}{ / }\PY{l+s}{\PYZdq{}}  \PY{o}{+} \PY{n+nb}{str}\PY{p}{(}\PY{n}{x}\PY{o}{.}\PY{n}{month}\PY{p}{)}\PY{p}{,} \PY{n}{labels}\PY{p}{)}
         \PY{n}{plt}\PY{o}{.}\PY{n}{xticks}\PY{p}{(}\PY{n}{t}\PY{p}{[}\PY{l+m+mi}{0}\PY{p}{:}\PY{p}{:}\PY{l+m+mi}{30}\PY{p}{]}\PY{p}{,} \PY{n}{date\PYZus{}str}\PY{p}{[}\PY{l+m+mi}{0}\PY{p}{:}\PY{p}{:}\PY{l+m+mi}{30}\PY{p}{]}\PY{p}{,} \PY{n}{rotation}\PY{o}{=}\PY{l+s}{\PYZdq{}}\PY{l+s}{vertical}\PY{l+s}{\PYZdq{}}\PY{p}{)}\PY{p}{;}
\end{Verbatim}

    \begin{center}
    \adjustimage{max size={0.9\linewidth}{0.9\paperheight}}{time_series_files/time_series_26_0.png}
    \end{center}
    { \hspace*{\fill} \\}

    \begin{Verbatim}[commandchars=\\\{\}]
{\color{incolor}In [{\color{incolor}15}]:} \PY{n}{det\PYZus{}curve} \PY{o}{=} \PY{n}{counts} \PY{o}{\PYZhy{}} \PY{n}{fit\PYZus{}curve}
         \PY{n}{plt}\PY{o}{.}\PY{n}{figure}\PY{p}{(}\PY{n}{figsize}\PY{o}{=}\PY{p}{(}\PY{l+m+mi}{25}\PY{p}{,} \PY{l+m+mi}{5}\PY{p}{)}\PY{p}{)}
         \PY{n}{plt}\PY{o}{.}\PY{n}{plot}\PY{p}{(}\PY{n}{det\PYZus{}curve}\PY{p}{,} \PY{n}{alpha}\PY{o}{=}\PY{l+m+mf}{0.6}\PY{p}{)}\PY{p}{;}
\end{Verbatim}

    \begin{center}
    \adjustimage{max size={0.9\linewidth}{0.9\paperheight}}{time_series_files/time_series_27_0.png}
    \end{center}
    { \hspace*{\fill} \\}

    \subsection{Further Confound Removal}\label{further-confound-removal}

Lets suppose that the quadratic is a measurement error and that the
$\approx$ 90 day sinusodial is a well understood or nuisance, then we'll
examine the remainder of the signal.

    \begin{Verbatim}[commandchars=\\\{\}]
{\color{incolor}In [{\color{incolor}16}]:} \PY{k+kn}{import} \PY{n+nn}{sys}
         \PY{n}{sys}\PY{o}{.}\PY{n}{path}\PY{o}{.}\PY{n}{append}\PY{p}{(}\PY{l+s}{\PYZdq{}}\PY{l+s}{/home/daniel/git/Python2.7/MRI/Modules}\PY{l+s}{\PYZdq{}}\PY{p}{)}
         \PY{k+kn}{import} \PY{n+nn}{SignalProcessTools}

         \PY{n}{sigtools} \PY{o}{=} \PY{n}{SignalProcessTools}\PY{o}{.}\PY{n}{SignalProcessTools}\PY{p}{(}\PY{p}{)}
\end{Verbatim}

    \subsubsection{Frequency Domain Analysis Using
FFT}\label{frequency-domain-analysis-using-fft}

I keep this method handy and it should be in my Sigtools Module. It's
just as well that you can see inside the Welch call.

    \begin{Verbatim}[commandchars=\\\{\}]
{\color{incolor}In [{\color{incolor}17}]:} \PY{k}{def} \PY{n+nf}{fft}\PY{p}{(}\PY{n}{data}\PY{p}{)}\PY{p}{:}
                 \PY{l+s+sd}{\PYZsq{}\PYZsq{}\PYZsq{}Plot FFT using Welch\PYZsq{}s method, daily resolution \PYZsq{}\PYZsq{}\PYZsq{}}
                 \PY{n}{f}\PY{p}{,} \PY{n}{y} \PY{o}{=} \PY{n}{signal}\PY{o}{.}\PY{n}{welch}\PY{p}{(}\PY{n}{data}\PY{p}{,} \PY{n}{fs}\PY{o}{=}\PY{l+m+mf}{1.0}\PY{p}{,} \PY{n}{nperseg}\PY{o}{=}\PY{l+m+mi}{365}\PY{p}{,} \PY{n}{noverlap}\PY{o}{=}\PY{n+nb+bp}{None}\PY{p}{,} \PY{n}{nfft}\PY{o}{=}\PY{l+m+mi}{512}\PY{p}{,} \PY{n}{scaling}\PY{o}{=}\PY{l+s}{\PYZsq{}}\PY{l+s}{spectrum}\PY{l+s}{\PYZsq{}}\PY{p}{,} \PY{n}{detrend}\PY{o}{=}\PY{l+s}{\PYZdq{}}\PY{l+s}{linear}\PY{l+s}{\PYZdq{}}\PY{p}{)}

                 \PY{n}{interval} \PY{o}{=} \PY{l+m+mi}{3} \PY{c}{\PYZsh{} days}
                 \PY{n}{periods} \PY{o}{=} \PY{n}{np}\PY{o}{.}\PY{n}{round}\PY{p}{(}\PY{l+m+mf}{1.}\PY{o}{/}\PY{n}{f}\PY{p}{[}\PY{l+m+mi}{0}\PY{p}{:}\PY{p}{:}\PY{n}{interval}\PY{p}{]}\PY{p}{,} \PY{l+m+mi}{1}\PY{p}{)}
                 \PY{c}{\PYZsh{} clean up frequency of 0 Hz}
                 \PY{n}{periods}\PY{p}{[}\PY{l+m+mi}{0}\PY{p}{]} \PY{o}{=} \PY{l+m+mi}{0}

                 \PY{n}{frqs} \PY{o}{=} \PY{n}{f}\PY{p}{[}\PY{l+m+mi}{0}\PY{p}{:}\PY{p}{:}\PY{n}{interval}\PY{p}{]}
                 \PY{n}{plt}\PY{o}{.}\PY{n}{xticks}\PY{p}{(}\PY{n}{frqs}\PY{p}{,} \PY{n}{periods}\PY{p}{,} \PY{n}{rotation}\PY{o}{=}\PY{l+s}{\PYZdq{}}\PY{l+s}{vertical}\PY{l+s}{\PYZdq{}}\PY{p}{)}

                 \PY{n}{plt}\PY{o}{.}\PY{n}{plot}\PY{p}{(}\PY{n}{f}\PY{p}{,} \PY{n}{y}\PY{p}{)}

                 \PY{n}{plt}\PY{o}{.}\PY{n}{grid}\PY{p}{(}\PY{n+nb+bp}{True}\PY{p}{)}
                 \PY{c}{\PYZsh{}plt.title(\PYZdq{}Welch FFT: Counts\PYZdq{})}
                 \PY{n}{plt}\PY{o}{.}\PY{n}{ylabel}\PY{p}{(}\PY{l+s}{\PYZdq{}}\PY{l+s}{Relative ratio of spectrum}\PY{l+s}{\PYZdq{}}\PY{p}{)}
                 \PY{n}{plt}\PY{o}{.}\PY{n}{xlabel}\PY{p}{(}\PY{l+s}{\PYZdq{}}\PY{l+s}{Number of days in a period}\PY{l+s}{\PYZdq{}}\PY{p}{)}

                 \PY{k}{return} \PY{n}{f}\PY{p}{,} \PY{n}{y}\PY{p}{,} \PY{n}{frqs}
\end{Verbatim}

    \begin{Verbatim}[commandchars=\\\{\}]
{\color{incolor}In [{\color{incolor}18}]:} \PY{n}{plt}\PY{o}{.}\PY{n}{figure}\PY{p}{(}\PY{n}{figsize}\PY{o}{=}\PY{p}{(}\PY{l+m+mi}{20}\PY{p}{,} \PY{l+m+mi}{5}\PY{p}{)}\PY{p}{)}
         \PY{n}{f}\PY{p}{,} \PY{n}{y}\PY{p}{,} \PY{n}{frqs} \PY{o}{=} \PY{n}{fft}\PY{p}{(}\PY{n}{det\PYZus{}curve}\PY{p}{)}
\end{Verbatim}

    \begin{center}
    \adjustimage{max size={0.9\linewidth}{0.9\paperheight}}{time_series_files/time_series_32_0.png}
    \end{center}
    { \hspace*{\fill} \\}

    \begin{Verbatim}[commandchars=\\\{\}]
{\color{incolor}In [{\color{incolor}19}]:} \PY{n}{frq} \PY{o}{=} \PY{l+m+mi}{1} \PY{o}{/} \PY{l+m+mf}{56.9} \PY{c}{\PYZsh{} from FFT output above}
         \PY{n}{out} \PY{o}{=} \PY{n}{sigtools}\PY{o}{.}\PY{n}{hi\PYZus{}pass\PYZus{}filter}\PY{p}{(}\PY{n}{det\PYZus{}curve}\PY{p}{,} \PY{n}{frq}\PY{p}{,} \PY{l+m+mf}{1.0}\PY{p}{,} \PY{l+m+mi}{3}\PY{p}{)}

         \PY{n}{dff} \PY{o}{=} \PY{n}{pd}\PY{o}{.}\PY{n}{DataFrame}\PY{p}{(}\PY{p}{\PYZob{}}\PY{l+s}{\PYZsq{}}\PY{l+s}{count}\PY{l+s}{\PYZsq{}}\PY{p}{:}\PY{n}{out}\PY{p}{\PYZcb{}}\PY{p}{,} \PY{n}{index}\PY{o}{=}\PY{n}{df}\PY{o}{.}\PY{n}{index}\PY{p}{)}
         \PY{n}{dff}\PY{o}{.}\PY{n}{plot}\PY{p}{(}\PY{n}{title}\PY{o}{=}\PY{l+s}{\PYZdq{}}\PY{l+s}{Counts After Quadratic Detrend and High Pass Filter}\PY{l+s}{\PYZdq{}}\PY{p}{,} \PY{n}{figsize}\PY{o}{=}\PY{p}{(}\PY{l+m+mi}{25}\PY{p}{,} \PY{l+m+mi}{6}\PY{p}{)}\PY{p}{,} \PY{n}{alpha}\PY{o}{=}\PY{l+m+mf}{0.6}\PY{p}{)}
\end{Verbatim}

            \begin{Verbatim}[commandchars=\\\{\}]
{\color{outcolor}Out[{\color{outcolor}19}]:} <matplotlib.axes.\_subplots.AxesSubplot at 0x7f9e9a6e5210>
\end{Verbatim}

    \begin{center}
    \adjustimage{max size={0.9\linewidth}{0.9\paperheight}}{time_series_files/time_series_33_1.png}
    \end{center}
    { \hspace*{\fill} \\}

    \subsection{Fit A Distribution}\label{fit-a-distribution}

The somewhat manual way and after with Seaborn

    \begin{Verbatim}[commandchars=\\\{\}]
{\color{incolor}In [{\color{incolor}20}]:} \PY{n}{plt}\PY{o}{.}\PY{n}{figure}\PY{p}{(}\PY{n}{figsize}\PY{o}{=}\PY{p}{(}\PY{l+m+mi}{8}\PY{p}{,} \PY{l+m+mi}{8}\PY{p}{)}\PY{p}{)}

         \PY{c}{\PYZsh{} undo the centering that occures fromt he previous procesing}
         \PY{n}{count} \PY{o}{=} \PY{n}{dff}\PY{p}{[}\PY{l+s}{\PYZsq{}}\PY{l+s}{count}\PY{l+s}{\PYZsq{}}\PY{p}{]} \PY{o}{\PYZhy{}} \PY{n}{dff}\PY{p}{[}\PY{l+s}{\PYZsq{}}\PY{l+s}{count}\PY{l+s}{\PYZsq{}}\PY{p}{]}\PY{o}{.}\PY{n}{min}\PY{p}{(}\PY{p}{)}

         \PY{n}{nbins} \PY{o}{=} \PY{n}{np}\PY{o}{.}\PY{n}{floor}\PY{p}{(}\PY{n}{np}\PY{o}{.}\PY{n}{sqrt}\PY{p}{(}\PY{n}{count}\PY{o}{.}\PY{n}{shape}\PY{p}{[}\PY{l+m+mi}{0}\PY{p}{]}\PY{p}{)}\PY{p}{)}
         \PY{n}{y\PYZus{}counts}\PY{p}{,} \PY{n}{bin\PYZus{}x}\PY{p}{,} \PY{n}{patch} \PY{o}{=} \PY{n}{plt}\PY{o}{.}\PY{n}{hist}\PY{p}{(}\PY{n}{count}\PY{p}{,} \PY{n}{nbins}\PY{p}{,} \PY{n}{normed}\PY{o}{=}\PY{n+nb+bp}{True}\PY{p}{,} \PY{n}{alpha}\PY{o}{=}\PY{l+m+mf}{0.5}\PY{p}{)}\PY{p}{;}
         \PY{n}{string} \PY{o}{=} \PY{l+s}{\PYZdq{}}\PY{l+s}{Histrogram With }\PY{l+s+si}{\PYZpc{}s}\PY{l+s}{ Bins}\PY{l+s}{\PYZdq{}} \PY{o}{\PYZpc{}} \PY{n+nb}{str}\PY{p}{(}\PY{n+nb}{int}\PY{p}{(}\PY{n}{nbins}\PY{p}{)}\PY{p}{)}
         \PY{n}{plt}\PY{o}{.}\PY{n}{title}\PY{p}{(}\PY{n}{string}\PY{p}{)}
\end{Verbatim}

            \begin{Verbatim}[commandchars=\\\{\}]
{\color{outcolor}Out[{\color{outcolor}20}]:} <matplotlib.text.Text at 0x7f9e9a4ac510>
\end{Verbatim}

    \begin{center}
    \adjustimage{max size={0.9\linewidth}{0.9\paperheight}}{time_series_files/time_series_35_1.png}
    \end{center}
    { \hspace*{\fill} \\}

    Perhaps a Gausian

    \begin{Verbatim}[commandchars=\\\{\}]
{\color{incolor}In [{\color{incolor}21}]:} \PY{n}{bin\PYZus{}x} \PY{o}{=} \PY{n}{bin\PYZus{}x}\PY{p}{[}\PY{l+m+mi}{1}\PY{p}{:}\PY{p}{]}\PY{c}{\PYZsh{} drop the first bin to match the array lengths}
\end{Verbatim}

    \begin{Verbatim}[commandchars=\\\{\}]
{\color{incolor}In [{\color{incolor}22}]:} \PY{k+kn}{from} \PY{n+nn}{scipy.optimize} \PY{k+kn}{import} \PY{n}{curve\PYZus{}fit}

         \PY{k}{def} \PY{n+nf}{gauss}\PY{p}{(}\PY{n}{x}\PY{p}{,} \PY{o}{*}\PY{n}{p}\PY{p}{)}\PY{p}{:}
             \PY{n}{A}\PY{p}{,} \PY{n}{mu}\PY{p}{,} \PY{n}{sig} \PY{o}{=} \PY{n}{p}
             \PY{n}{gau} \PY{o}{=} \PY{n}{A} \PY{o}{*} \PY{n}{np}\PY{o}{.}\PY{n}{exp}\PY{p}{(}\PY{o}{\PYZhy{}}\PY{p}{(}\PY{n}{x}\PY{o}{\PYZhy{}}\PY{n}{mu}\PY{p}{)}\PY{o}{*}\PY{o}{*}\PY{l+m+mi}{2} \PY{o}{/} \PY{p}{(}\PY{l+m+mi}{2} \PY{o}{*} \PY{n}{sig}\PY{p}{)}\PY{o}{*}\PY{o}{*}\PY{l+m+mi}{2}\PY{p}{)}
             \PY{k}{return} \PY{n}{gau}
\end{Verbatim}

    \begin{Verbatim}[commandchars=\\\{\}]
{\color{incolor}In [{\color{incolor}23}]:} \PY{n}{coeff}\PY{p}{,} \PY{n}{var\PYZus{}matrix} \PY{o}{=} \PY{n}{curve\PYZus{}fit}\PY{p}{(}\PY{n}{gauss}\PY{p}{,} \PY{n}{bin\PYZus{}x}\PY{p}{,} \PY{n}{y\PYZus{}counts}\PY{p}{,} \PY{n}{p0}\PY{o}{=}\PY{p}{[}\PY{l+m+mf}{0.001}\PY{p}{,} \PY{l+m+mf}{0.0001}\PY{p}{,} \PY{l+m+mf}{20.0}\PY{p}{]}\PY{p}{)}
         \PY{k}{print} \PY{l+s}{\PYZdq{}}\PY{l+s}{Amplitude:}\PY{l+s+si}{\PYZpc{}f}\PY{l+s}{  mean:}\PY{l+s+si}{\PYZpc{}f}\PY{l+s}{   std:}\PY{l+s+si}{\PYZpc{}f}\PY{l+s}{\PYZdq{}} \PY{o}{\PYZpc{}}\PY{p}{(}\PY{n}{coeff}\PY{p}{[}\PY{l+m+mi}{0}\PY{p}{]}\PY{p}{,} \PY{n}{coeff}\PY{p}{[}\PY{l+m+mi}{1}\PY{p}{]}\PY{p}{,} \PY{n}{coeff}\PY{p}{[}\PY{l+m+mi}{2}\PY{p}{]}\PY{p}{)}
\end{Verbatim}

    \begin{Verbatim}[commandchars=\\\{\}]
Amplitude:0.020397  mean:79.954072   std:13.724083
    \end{Verbatim}

    \begin{Verbatim}[commandchars=\\\{\}]
{\color{incolor}In [{\color{incolor}24}]:} \PY{n}{fit\PYZus{}gau} \PY{o}{=} \PY{n}{gauss}\PY{p}{(}\PY{n}{bin\PYZus{}x}\PY{p}{,} \PY{o}{*}\PY{n}{coeff}\PY{p}{)}
         \PY{n}{plt}\PY{o}{.}\PY{n}{figure}\PY{p}{(}\PY{n}{figsize}\PY{o}{=}\PY{p}{(}\PY{l+m+mi}{7}\PY{p}{,} \PY{l+m+mi}{7}\PY{p}{)}\PY{p}{)}

         \PY{n}{plt}\PY{o}{.}\PY{n}{plot}\PY{p}{(}\PY{n}{bin\PYZus{}x}\PY{p}{,} \PY{n}{fit\PYZus{}gau}\PY{p}{,} \PY{l+s}{\PYZsq{}}\PY{l+s}{r}\PY{l+s}{\PYZsq{}}\PY{p}{)}
         \PY{n}{plt}\PY{o}{.}\PY{n}{bar}\PY{p}{(}\PY{n}{bin\PYZus{}x}\PY{p}{,} \PY{n}{y\PYZus{}counts}\PY{p}{,} \PY{n}{alpha}\PY{o}{=}\PY{l+m+mf}{0.4}\PY{p}{,} \PY{n}{width}\PY{o}{=}\PY{l+m+mi}{3}\PY{p}{,} \PY{n}{align}\PY{o}{=}\PY{l+s}{\PYZsq{}}\PY{l+s}{center}\PY{l+s}{\PYZsq{}}\PY{p}{)}
\end{Verbatim}

            \begin{Verbatim}[commandchars=\\\{\}]
{\color{outcolor}Out[{\color{outcolor}24}]:} <Container object of 42 artists>
\end{Verbatim}

    \begin{center}
    \adjustimage{max size={0.9\linewidth}{0.9\paperheight}}{time_series_files/time_series_40_1.png}
    \end{center}
    { \hspace*{\fill} \\}

    \subsection{Jackknife Bias Estimation}\label{jackknife-bias-estimation}

Since the time dependant trends have been removed, the remaining records
should be independent identically distributed records, or close to it.

We are putting a lot of weight ont he Guassian fit above. That mean and
standard deviation are the main metrics that describe the data. To find
out if our method of computing the mean and sigma are biased, we can use
a resampling technique. The most simple is the Jackknife.

    \begin{Verbatim}[commandchars=\\\{\}]
{\color{incolor}In [{\color{incolor}25}]:} \PY{c}{\PYZsh{}\PYZsh{} Helper functions for the jackknife iterations }

         \PY{k}{def} \PY{n+nf}{jk\PYZus{}gau\PYZus{}fit}\PY{p}{(}\PY{n}{sample}\PY{p}{)}\PY{p}{:}
             \PY{n}{nbins} \PY{o}{=} \PY{n}{np}\PY{o}{.}\PY{n}{floor}\PY{p}{(}\PY{n}{np}\PY{o}{.}\PY{n}{sqrt}\PY{p}{(}\PY{n}{sample}\PY{o}{.}\PY{n}{shape}\PY{p}{[}\PY{l+m+mi}{0}\PY{p}{]}\PY{p}{)}\PY{p}{)}
             \PY{n}{y\PYZus{}counts}\PY{p}{,} \PY{n}{bin\PYZus{}x} \PY{o}{=} \PY{n}{np}\PY{o}{.}\PY{n}{histogram}\PY{p}{(}\PY{n}{sample}\PY{p}{,} \PY{n}{nbins}\PY{p}{,} \PY{n}{normed}\PY{o}{=}\PY{n+nb+bp}{True}\PY{p}{)}\PY{p}{;} \PY{c}{\PYZsh{} slightly different method is faster as there is no plotting}
             \PY{n}{bin\PYZus{}x} \PY{o}{=} \PY{n}{bin\PYZus{}x}\PY{p}{[}\PY{l+m+mi}{1}\PY{p}{:}\PY{p}{]}
             \PY{n}{coeff}\PY{p}{,} \PY{n}{var\PYZus{}matrix} \PY{o}{=} \PY{n}{curve\PYZus{}fit}\PY{p}{(}\PY{n}{gauss}\PY{p}{,} \PY{n}{bin\PYZus{}x}\PY{p}{,} \PY{n}{y\PYZus{}counts}\PY{p}{,} \PY{n}{p0}\PY{o}{=}\PY{p}{[}\PY{l+m+mf}{0.001}\PY{p}{,} \PY{l+m+mf}{0.0001}\PY{p}{,} \PY{l+m+mf}{20.0}\PY{p}{]}\PY{p}{)}

             \PY{k}{return} \PY{n}{coeff}\PY{p}{[}\PY{l+m+mi}{2}\PY{p}{]} \PY{c}{\PYZsh{} amp, mean, sig}

         \PY{k}{def} \PY{n+nf}{jk\PYZus{}params}\PY{p}{(}\PY{n}{data}\PY{p}{)}\PY{p}{:}
             \PY{n}{sig} \PY{o}{=} \PY{n}{np}\PY{o}{.}\PY{n}{zeros\PYZus{}like}\PY{p}{(}\PY{n}{data}\PY{p}{)}
             \PY{n}{n} \PY{o}{=} \PY{n}{data}\PY{o}{.}\PY{n}{shape}\PY{p}{[}\PY{l+m+mi}{0}\PY{p}{]}
             \PY{k}{for} \PY{n}{i} \PY{o+ow}{in} \PY{n+nb}{range}\PY{p}{(}\PY{n}{n}\PY{p}{)}\PY{p}{:}
                 \PY{n}{sample} \PY{o}{=} \PY{n}{np}\PY{o}{.}\PY{n}{delete}\PY{p}{(}\PY{n}{data}\PY{p}{,} \PY{n}{i}\PY{p}{)}
                 \PY{n}{sig}\PY{p}{[}\PY{n}{i}\PY{p}{]} \PY{o}{=} \PY{n}{jk\PYZus{}gau\PYZus{}fit}\PY{p}{(}\PY{n}{sample}\PY{p}{)}

             \PY{k}{return} \PY{n}{sig}
\end{Verbatim}

    \begin{Verbatim}[commandchars=\\\{\}]
{\color{incolor}In [{\color{incolor}27}]:} \PY{n}{count} \PY{o}{=} \PY{n}{np}\PY{o}{.}\PY{n}{array}\PY{p}{(}\PY{n}{count}\PY{p}{)} \PY{c}{\PYZsh{} cast from dataframe/series into numpy array}
         \PY{n}{jksig} \PY{o}{=} \PY{n}{jk\PYZus{}params}\PY{p}{(}\PY{n}{count}\PY{p}{)}
\end{Verbatim}

    \subsubsection{Bias Calculation}\label{bias-calculation}

    \begin{Verbatim}[commandchars=\\\{\}]
{\color{incolor}In [{\color{incolor}43}]:} \PY{n}{n} \PY{o}{=} \PY{n}{count}\PY{o}{.}\PY{n}{shape}\PY{p}{[}\PY{l+m+mi}{0}\PY{p}{]}
         \PY{n}{sig\PYZus{}bias} \PY{o}{=}  \PY{p}{(}\PY{n}{n} \PY{o}{\PYZhy{}} \PY{l+m+mf}{1.0}\PY{p}{)}\PY{o}{/}\PY{n}{n} \PY{o}{*} \PY{n}{coeff}\PY{p}{[}\PY{l+m+mi}{2}\PY{p}{]} \PY{o}{\PYZhy{}} \PY{n}{jksig}\PY{o}{.}\PY{n}{mean}\PY{p}{(}\PY{p}{)}
\end{Verbatim}

    \begin{Verbatim}[commandchars=\\\{\}]
{\color{incolor}In [{\color{incolor}45}]:} \PY{c}{\PYZsh{} 64 bit floats have 32 sig figs}
         \PY{k}{print} \PY{l+s}{\PYZdq{}}\PY{l+s}{Original STD from Fit :   }\PY{l+s+si}{\PYZpc{}.4f}\PY{l+s}{\PYZdq{}}    \PY{o}{\PYZpc{}}\PY{k}{coeff}[2]
         \PY{k}{print} \PY{l+s}{\PYZdq{}}\PY{l+s}{Mean STD from Samples :   }\PY{l+s+si}{\PYZpc{}.4f}\PY{l+s+se}{\PYZbs{}n}\PY{l+s}{\PYZdq{}}  \PY{o}{\PYZpc{}}\PY{k}{jksig}.mean()
         \PY{k}{print} \PY{l+s}{\PYZdq{}}\PY{l+s}{Bias:                     }\PY{l+s+si}{\PYZpc{}07.4f}\PY{l+s}{\PYZdq{}} \PY{o}{\PYZpc{}}\PY{p}{(}\PY{n}{sig\PYZus{}bias}\PY{p}{)}
\end{Verbatim}

    \begin{Verbatim}[commandchars=\\\{\}]
Original STD from Fit :   13.7241
Mean STD from Samples :   13.3935

Bias:                     00.3230
    \end{Verbatim}

    Update the standard deviation for the Guassian Fit.

    \begin{Verbatim}[commandchars=\\\{\}]
{\color{incolor}In [{\color{incolor}46}]:} \PY{n}{sig} \PY{o}{=} \PY{n}{coeff}\PY{p}{[}\PY{l+m+mi}{2}\PY{p}{]} \PY{o}{\PYZhy{}} \PY{n}{sig\PYZus{}bias}

         \PY{k}{print} \PY{l+s}{\PYZdq{}}\PY{l+s}{sig:}\PY{l+s+si}{\PYZpc{}.1f}\PY{l+s}{\PYZdq{}} \PY{o}{\PYZpc{}}\PY{k}{sig} \PYZsh{} rounding to three sig figs.
\end{Verbatim}

    \begin{Verbatim}[commandchars=\\\{\}]
sig:13.4
    \end{Verbatim}

    The mean of the count taken to 3 sig figs.

$\mu_{count} = 80.0 \pm 13.4$

    \begin{Verbatim}[commandchars=\\\{\}]
{\color{incolor}In [{\color{incolor} }]:}
\end{Verbatim}


    % Add a bibliography block to the postdoc



    \end{document}
